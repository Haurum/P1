\chapter{Diskussion}
I dette kapitel vil gruppen opsamle på de beslutninger, som er blevet truffet gennem projektet.  Samtidig vil gruppen se på fejlkilder der kunne være opstået og hvilken indflydelse disse fejlkilder har kunne påvirke projektet. 
\section{Spørgeskema}
Tidligt i projektet sendte gruppen et spørgeskema ud på deres facebookprofiler, som i alt gav 60 besvarelser, og gennemsnits alderen ville være relativ lav. Gruppen havde en række spørgsmål som vi gerne ville have svar på, og derfor blev spørgeskemaet opbygget af mange underspørgsmål, som var blevet lavet i forhold til den initierende problemstilling. Spørgeskemaet konstaterede nogle problemer, hvor det nok var forudsigeligt hvad respondenterne ville svare på de stillede spørgsmål, dette var bl.a. om turisterne havde et problem med at finde rundt på deres storbyferie.

Turister var sat som gidsler i interessent analysen, men gruppen valgte at bruge dem som en delvis ressourceperson, da det var dem som programmet var rettet imod. Dog mener gruppen at spørgeskemaet blev sendt for hurtigt ud, og uden eftertanke. Med det menes det at mange af spørgsmålene ikke var formuleret ordenligt, og man let kunne gætte sig til hvad respondenterne ville svare på spørgsmålene, dermed at mange af spørgsmålene var meget ledende. Selve projektet er rettet mod turister i Aalborg, hvor 41\% af turisterne er udenlandske, så da gruppen havde valgt at skrive spørgeskemaet på dansk, rammes hele målgruppen ikke. Gruppen havde som sagt delt spørgeskemaet på deres facebookprofiler, hvilket også begrænser respondentgruppen, dog vil de fleste være dansktalende, hvilket i denne situation ikke gjorde så meget at spørgeskemaet så var skrevet på dansk. For at få flere respondenter, og specielt udenlandske respondenter, burde gruppen have delt spørgeskemaet flere steder end bare på Facebook, samt lavet spørgeskemaet på både engelsk og dansk. 

Når gruppen kigger tilbage på dette projekt forløb, ville vi gerne have brugt mere energi til på dette spørgeskema, og med eftertanke burde vi nok have udsendt et nyt spørgeskema. Selve spørgsmålene skulle være bedre gennemtænk og ikke så ledende som de var blevet skrevet. Ud fra spørgeskemaet blev der dog konstateret et problem, som projektet kunne tage udgangspunkt i. En respondent gruppe er dog relativ lille, og det ville have været bedre at få en både større men også bredere respondent gruppe. Med ordet bredere mener gruppen at der ønskes både respondenter fra ind- og udland i forskellige aldre.
\section{Interview}
I dette projektets interessentanalyse blev turistbureauer, i dette tilfælde VisitAalborg, sat som ressourceperson, da de ville kunne give gruppen en del informationer om turisme i Aalborg. Derfor besluttede gruppen på at prøve at skaffe et interview med en fra VisitAalborg, hvor gruppen så fik fat i Lars Bech, som gerne ville stille op til et interview. I gruppen havde vi lavet en interviewguide, hvilket var lavet i punktform i form af hvad vi ville fortælle om vores emne, og hvilke spørgsmål vi gerne ville stille til Lars. Interviewet var planlagt til at det skulle udføres som et ustruktureret interview, hvilket vil betyde at spørgsmålenes rækkefølge ikke er fastlagte, hvilket vil gøre et interview mere fleksibelt. Hvis Lars ikke var meget for at snakke, eller ikke kom frem med lige det vi kiggede efter, ville vi kunne spørge mere ind til emnet. Dette var dog ikke tilfældet med Lars, han snakkede rigtig meget, hvor han ofte førte interviewet videre. Da vi i gruppen ikke rigtig havde lavet interview før, var interviewerne ikke så gode til at stoppe ham, når han snakkede videre end de stillede spørgsmålene. Dette gjorde at interviewet udviklede sig til, at Lars nærmest tog styringen af interviewet.

\textbf{Dette er ikke færdigt endnu}
\section{Programmet}
Programmet er lavet som beskrevet i afsnittet ”Implementering”, og havde helt fra begyndelsen valgt at lave løsningen i fugleflugtslinje, og gruppen har derved ikke taget højde for vejnettet i Aalborg. Dette har simplificeret programmet, men også givet en usikkerhed når den hurtigste rute skal bestemmes. Gruppen kan ikke garantere, at den givne rute i realiteten er den hurtigste, når der også skal tages højde for hvilke veje man rent faktisk kan bevæge sig på og kan komme igennem. 
Hvis gruppen skulle have implementeret en løsning, der tager højde for vejnettet, havde gruppen tænkt på to forskellige løsninger. Den første ville være at sætte hele Aalborg op i et grid, hvor vejnettet ville blive markeret med 1 og resten ville markeret med 0. På den måde ville vejen kunne findes med forskellige søgealgoritmer fx A*. En anden løsning ville være at have en tabel, der indikerede hvilke veje der er forbundet og distancen der i mellem. Disse løsninger ville have gjort ruten mere præcis, da den reelle korteste rute ville kunne findes. 

