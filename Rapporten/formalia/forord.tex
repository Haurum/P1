{\Huge\textbf{Forord}}

Dette projekt er udarbejdet i et samarbejde mellem syv software-studerende på første semester fra Det Teknisk-Videnskabelige Fakultet på Aalborg Universitet. 

I udarbejdelsen af projektet, har gruppen taget udgangspunkt i Aalborg-modellen, i form af problem- og projektbaseret læring. Der tages udgangspunkt i et problem, hvor læringen sker i form af projektarbejde i grupper.

Gruppen vil gerne sige tak til hovedvejlederen Jane Billestrup og bivejleder Mona-Lisa Dahms, for deres vejledning gennem projektet. Derudover vil gruppen også gerne takke Lars Bech og Kim Mikael Jensen fra VisitAalborg, for at stille op til et interview med gruppen. \newline
\newline
\newline
{\Huge\textbf{Læsevejledning}}

{\Large\textbf{Kildehenvisning}}\newline
I dette projekt bruges Havard-metoden, også kendt som Chicago-metoden, til kilde henvisning. Hvis der henvises til en bestemt kilde, efter eksempelvis en sætning, påstand eller et citat, henvises der på følgende måde: Sætning/påstand/citat [Forfatter, udgivelsesår].\newline
Hvis kilden anvendes til hele afsnit, sættes kilde henvisningen efter punktummet, så ledes: Afsnit.[Forfatter, udgivelsesår]\newline
I afsnittet ”Litteratur” vil kilde henvisningerne blive sorteret i alfabetiskrækkefølge, hvis det eksempelvis var en hjemmeside der blev brugt som kilde, ville det se så ledes ud: \newline
\textbf{Kilde henvisningen fra rapporten}. Forfatter. \textit{Titel}. URL. Udgivelsesår. Dato siden er set og evt. side tal.

\textbf{Dansk Statistik, 2008}. Dansk Statistik. \textit{Turismen – Regionalt, nationalt og internationalt}. http://www.dst.dk/pukora/epub/upload/11676/tur08.pdf, 2008. Set d. 19/11-2014 – side 8.

Hvis nogle af disse informationer mangler, eksempelvis udgivelsesår, udelades de.

{\Large\textbf{Figurhenvisning}} \newline
I gennem rapporten vil der blive henvist til figurer og illustrationer, hvor det vil blive anvist ud fra hvilket afsnit det befinder sig i, samt hvilket nummer i figuren er i det omtalte afsnit. Herudover skal der være en beskrivende tekst, der forklarer figuren, eksempelvis:
\begin{flushleft}
{\LARGE\textbf{Figur 2, afsnit 5: Figurtekst}}\newline
Figur 5.2: Figurbeskrivelse
\end{flushleft}
