\section{Begreber}
Ved en ’interessant rute’ vil der i denne rapport refereres til en rute, der indkluderer unikke kulturelle og nationale oplevelser, hvor det interessante aspekt er individuelt for hver turist. Steder der ville kunne gøre ruten mere interessant vil bl.a. være havnefronten, gågaden eller andre attraktioner turisten ikke selv havde tænkt på, som turisten ville kunne passere på sin vej fra A til B.\newline

En ’attraktion’ vil i denne rapport forstås som både traditionelle, og utraditionelle attraktioner. Traditionelle attraktioner i Aalborg er Aalborg Zoo, kunstmuseet og lignende. De utraditionelle indkluderer steder som Storcenteret, Ikea og Havnefronten.\newline

’Turist’ vil i denne rapport være til dels defineret ud fra UNWTOs (UN World Tourism Organistation) beskrivelse, som beskriver to slags turister. En-dagsturister, som maksimalt overnatter én nat på stedet, og en generel turist, som overnatter mere end en nat\citep{Turismen}. Udover disse to typer findes undergruppen ’erhvervsturister’. Dette er turister, der kommer til stedet med et arbejdsformål. \newline
En turist er en person, der har et bestemt formål med rejsen, til forskel fra besøgende og rejsende. Formålet kan fx være, at opleve den danske kultur. \newline

En ’storby’ i Danmark, er i denne rapport en by, omgivet af områder med lavere bebyggelse. Hvor Aalborg fx har Nørresundby og Vejgaard.

Med dette afsnit har gruppen dannet grundlag for analysen, hvori de beskrevne begreber vil blive brugt. I de kommende afsnit vil den initierende problemstilling og dets spørgsmål blive analyseret.