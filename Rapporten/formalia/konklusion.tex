\chapter{Konklusion}
I dette kapitel vil der blive konkluderet på, om løsningen besvarer den opstillede problemformulering fra problembeskrivelsen.

Problemformulering:\newline
\textit{"Hvordan udvikles der en softwareløsning, der hjælper turisten med at finde rundt i en storby, på en interessant rute mellem turistens egne valgte attraktioner?"}

Igennem processen af problemløsningen, blev der udviklet et stykke software, der hjælper turisten med at finde en interessant rute, i fugleflugtslinje. Denne fugleflugtslinje er der ikke et kort over, hvilket betyder, at turisten ikke bliver hjulpet i et særlig stort omfang. Hjælpen fra denne softwareløsning er blot et forslag for, hvilke attraktioner der skal besøges, og i hvilken rækkefølge med kortest mulig distance mellem disse. Den interessante rute udgøres af de attraktioner brugeren vælger, samt de attraktioner der er mulighed for at tilføje, når de bliver spurgt om yderligere attraktioner til deres rute, hvis der findes attraktioner tæt på deres nuværende rute.\newline
Et optimalt hjælpemiddel til at finde rundt i en storby ville være en løsning der kortlægger ruten, hvilket dette program ikke har formået. En eventuel løsning på dette, ville være at sætte attraktionerne ind på et kort, ved hjælp af de allerede benyttede koordinater. Problemet er at finde et kort der fungere i C, og ikke fx JavaScript som Google Maps benytter.

Der kan tilnærmelsesvis siges, at det software der i denne rapport er programmeret stadig hjælper turisten med at finde rundt, da den giver en liste med rækkefølgen over attraktionerne, som brugeren har bestemt. Denne løsning kan ikke vise vej uden at brugeren selv indtaster attraktionerne i fx Google Maps. Programmet kan hverken kortlægge eller guide, men fortæller udelukkende om rutens planlagte rækkefølge.