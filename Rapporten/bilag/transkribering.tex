\section{Transskribering}
\textbf{Mark:} Vi fik et oplæg om at man kan bruge Google Maps til at finde en rute fra et punkt til et andet, men man har ikke muligheden for at angive mange punkter og få den bedste rute. Så det vi egentlig har fået til opgave, er at lave et sådan produkt, der kan det, og i den anledning tænkte vi at turister kunne værre en case. Sådan noget med at finde rundt mellem seværdigheder, finde ud af hvad de vil se, og så give dem en rute der kan vise dem rundt i Aalborg, eller enhver storby, men nu var det lige Aalborg vi havde fokus på. Så det er egentlig meget generelt det. Så har vi haft nogle spørgeskemaundersøgelser, hvor vi har spurgt folk om hvad de gerne vil se, og om de vil have fokus på at se det mest interessante ting i Aalborg, eller bare vil have en hurtig tur rundt. De svarede så, at de gerne ville have den mest interessante rute. \newline
\textbf{Mikael:}Hvis de vidste at de skulle fra et punkt til et andet, eller den ene attraktion til den anden, om de så vil have nogle ideer til hvad der er den mest interessante vej, altså måske skulle de gå en km ekstra, men så fik de set nogle andre små attraktioner, eller om de bare ville have den hurtigste rute, hvor vi kom frem til at folk gerne vil have den mest interessante rute, hvor de kan krydse af hvad de vil ud og se. \newline
\textbf{Mark:} Ideen for os, ville så være at lave en lille app, man har på sin telefon, hvor man så kan se at folk fx har ranked Aalborg Tårnet til at være 4/5, ”har du så lyst til at se den” hvis folk kommer forbi. Det kunne værre sådan noget i den stil. Ja, så det vi først gerne vil høre noget om, er turisme i Aalborg. \newline
\textbf{Lars:} Til alt det specifikke der, der henter jeg lige min kollega Kim, hvis han lige har tid, for der er jeg fuldstændig blank på viden om hvad vi lige gør, så lad os tage de der generelle spørgsmål først, og når vi så bliver lidt mere specifikke, så finder jeg lige ud af hvilken ven jeg skal ringe til. \newline
\textbf{Mark:} Godt, jamen turisme i Aalborg. Hvordan, hvorledes, hvor mange og hvad vil de gerne se, og hvordan hjælper i dem rundt? \newline
\textbf{Lars:} Altså, vi måler det jo på flere forskellige måder. Den mest anvendte, det er at se hvor mange overnatninger vi har i byen. Det ved vi jo fra byens hoteller, som indberetter det til Horesta, som så gør det tilgængelig hos Danmark Statistik, så vi kan gå tilbage og se hvor mange overnatninger der har været i juni måned 2014, og hvordan så det ud i forhold til juni måned 2013, og hvordan var fordeling på nationaliteter, var der fremgang på norske, var der tilbagegang på svenske, hvad med engelske, kinesiske osv. Osv. Så det er den måde vi normalt pejler efter, når vi snakker om hvorvidt det går godt eller skidt for turisme i Aalborg. \newline
Men der er så lidt en ubekendt faktor, som vi ikke har nogle præcise tal på. Det er alle de gæster som kommer uden at overnatte. Det er fx de gæster der tager færgen oppe fra Kristiansand til Hirtshals og styre deres automobil til Bilka, hvor de voldboller supermarkedet derinde, fylder bagagerummet op med kølevare og hvad de eller lige har brug og behov for, og som i øvrigt er møg hamrende billigt, for i Norge koster alt 2-3 gange så meget som det gør i Danmark, og så går de typisk jo ind i midtbyen, eller i en af restauranterne derude, shopper lidt rundt og hygger sig lidt, inden de så om aftenen kører tilbage igen og tager færgen hjem. Dem har vi jo ikke så forfærdelig mange tal på. Vi ved hvor mange endagsturister Color Line har med færgen, men vi ved rent faktisk ikke om de kører til Hjørring, eller kører til Aalborg eller til Århus. Vi ved af erfaring og fornemmelse, at rigtig mange kører til Aalborg fordi en af de primære grunde til at de kører hertil, er Bilka. Og sådan er der selvfølgelig også nogle andre målgrupper, hvor man kan sige der er nogle der falder uden for den der målbare ramme. Og det har vi ikke rigtig fundet ud af at få sat i system endnu. Hvor mange turister er der? Jamen jeg tror lige jeg vil tage at vise jer nogle nøgletal. \newline
\textbf{Lars går ud og henter nogle brochurer:} Se nu får i lige en lille gave. Se det der er sådan set vores partnerkoncept. Det er det materiale vi bruger til at gå ud og erhvervet til at forstå at vi ligesom står sammen om den her markedsføring. VisitAalborg er jo sådan delvist kommunalt ejet. Det er kommunen der betaler den største del af vores drift. Lige knap 40% af de knap 15 millioner vi har i årlig driftsbudget, de er finansieret af kommunen. Resten, dem skaffer vi sådan set selv, ved hjælp af partnerskaber med erhvervet, hoteller, restauranter, attraktioner, detailhandelskæder, virksomheder, lufthavn, kongres og kulturcenter etc. etc. og desuden så gør vi også det at vi søger samarbejde både i regionen, med andre aktor uden for Aalborg, om at pulje pengene for at få så meget at skyde med som overhovedet muligt, men vi har også folk der sidder og arbejder på og lave projekt ansøgninger, forslag til de forskellige ministerielle og EU relaterede puljer, som kan give medfinansiering til nogle spændende og visionære projekter. Og det vi tilbyder ? det er sådan set delt op i et partnerkoncept som dels er basispartnerskab så er der nogle lidt dyrere business partnerskaber og VIP-partnerskaber som betaler 75.000kr om året, jeg skal ikke trætte jer med hvad det er, man kan ren faktisk læse det her i hvis i synes det er interessant hvad det er vi tilbyder, men nøgletallene som vi hviler på når vi skal definere os selv og vigtigheden af det vi gør, de står her nedenunder. I 2012 som er de seneste tilgængelige tal på omsætningen, ja der generede Aalborg en omsætning på lig knap 3mia kr. I relation til turister. En halv mio. til overnatninger, heraf det sådan set meget markant viser os hvor stor markedet det er. Heraf havde vi 140.000, ikke gæster, men roomnights, altså værelser solgt til norske gæster. Dertil skal så lægges de her norske endagsturister, som der også er rigtige mange af. Så hvis forbindelsen mellem Hirtshals og Norge eller Frederikshavn og Norge, den lige pludselige bliver oversvømmet eller afbrudt, så ville det betyde massive butiks? I Aalborg + at serviceerhverv virkelige komme på røven. Det kan man jo også se fordi at turismen alene står bag 4000 fuldtidsjob baseret på 2012 til ikke os?? Så vi er store , vier stærke og VisitAalborg arbejder, synes vi selv, top professionelt og i øvrigt Danmarks næststørste turisme-fremmeorganisation, der er kun København, Wonderfulcopenhagen som er vores kolleager i København, der er kun dem der er større, vi er mange gange større end de er i Århus og i Odense og de byer som vi ellers skulle normalt sammenligne os med. Så da vi er rent faktisk også gjort noget for ligesom ikke bare fastholde, men også udvikle tiltrækningen af flere gæster, ja, og så er der nogle andre nøgleting som kan have sine relevans. \newline
\textbf{Mikael:} Hvad med den der vækst der? \newline
\textbf{Lars:} Ja. \newline
\textbf{Mikael:} Er det, ved i hvad det sådan skyldes, eller er det kun i Aalborg eller for hele? \newline
\textbf{Lars:} Det jo en vækst i turisme og omsætning. \newline
\textbf{Mikael:} Ja. \newline. 
\textbf{Lars:} Det jo en vækst i turisme og omsætning som i 2012 var på 2,9mia. ikke. Der har man så regnet ud at i 2011, der var den så 4,6% lavere, så det sådan set bare en omskrivning af det, men det her det er ligesom for at det skal gerne virke som en imponator ikke, så folk siger hold da op, det er en fin vækst der må der være nogen ting vi har gjort rigtig. Så det er i forhold til året før. \newline
\textbf{Mikael:} Ved i hvad det skyldes? \newline
\textbf{Lars:} Ja, vi har fået noget mere løn, nej, det ved vi ikke, Jeg ved det ikke i hvert fald. Jeg har ikke dyrket tallene i nøjagtighed, men det skyldes jo selvfølgelig forskellige ting, ikke også, det sjældent at man sådan kan finde en grund til noget som helst. \newline
\textbf{Mikael:} Men jeg tænkte, når du siger med Norge at man så hvis man nu har fået bedre transport eller bedre skib fx at der kom de der 2 timers ture? \newline
\textbf{Lars:} 3 timers, ja. \newline
\textbf{Mikael:} Ja, 3 timers. \newline
\textbf{Lars:} Godt tænkt. \newline
\textbf{Mikael:} Så ved man ligesom.\newline
\textbf{Lars:} Ja, godt tænkt. \newline
\textbf{Mikael:} Så stiger..?\newline
\textbf{Lars:} Ja det ender med at vi bliver nød til at pulse. \newline
\textbf{Mikael:} der må man sikkert afsted meget. \newline
\textbf{Lars:} Ja, ja, det har du sku helt ret, du kender lidt til det også kan jeg høre. \newline
\textbf{Mikael:} Ja, jeg har taget det tog mange, eller jeg har taget den færge mange gange. \newline
\textbf{Mark:} Han er nordmand. \newline
\textbf{Lars:} Er du det? \newline
\textbf{Mikael:} Hmm.. \newline
\textbf{Efter en lang snak om noget irrelevant, spørger Lars:} Hvad var det egentlig for et spørgsmål du stillede? \newline
\textbf{Mark:} Har i også sådan noget guidet ture? \newline
\textbf{Lars:} Ja, det har vi. VisitAalborg er jo et kommunalt foretagende, så vi skal passe meget på med ikke at gå ud og lave, hvad skal man sige, konkurrenceforvridende virksomhed, så vi søsætter egentlig ikke selv guidede ture, men vi gør alt hvad vi kan for at understøtte folk som tager nogle initiativer i retning af at skabe nogen gæsteoplevelser og det næsten uanset i hvilken retning det er, vi deltager gerne både med råd og vejledning og nogle gange hjælper vi også hvis vi kan komme afsted med det med økonomi til nogen event der er gæsterelateret og det uanset om det er til Aalborg eller det til ”Forstår ikke.. hvilket race?” eller hvad det nu måtte være, eller det på den anden side er til erhvervsturisme, altså møder, kongresser, konferencer etc. Og desuden så er vi jo  også som en seriøs og professionel operatør, optaget af at der bliver lavet nogle nye tiltage, at der bliver udviklet nogle nye muligheder som kan være med til at forbedre gæsteoplevelsen og det er uanset om det er guidet cykelture, eller kørsel med åben top-bus eller hvad det nu måtte være, vi har alle sammen spillet et rolle for at få det etableret, problemet er at i modsætning til København som har et noget større fundament at stå på med hensyn til antal gæster i byen, så er det lidt vanskeligere at etablere i en by som Aalborg fordi vi lige måske er niveauet under, der hvor det er nemt at få det til at blive til en rigtig god forretning, men det så på den anden side er der hvor vi så gør en stor indsats for at være med til at skabe det bedst mulige fundament for at de projekter vi så hjælper på vej også for den bedste chance for at skabe deres eget forretningsgrundlag, så det går vi aktivt ind og hjælper i, men det er ikke os der står som afsender at alt det, det må vi for det første ikke og for det andet så kan vi ikke, men vi finder nogle lokale aktører og arbejder sammen med fx i tilfælde med åben top bus, der arbejder vi sammen med et lokalt busfirma som vi hjalp med at finansiere indkøbet af sådan en åben top-bus og så kører den så 3 gange om dagen de 8 sommerferie uger, det så i øvrigt uden for sæsonen brugt til nogle andre formål, så der kunne laves en business case der hang nogenlunde sammen. \newline
\textbf{Mark:} Attraktioner i Aalborg, hvad er der interessant at se? Hvad er der mange der tager til? \newline
\textbf{Lars:} Ja, det er jo forholdsvis enkelt hvis man bruger det traditionelle. Hvis man bruger det utraditionelle, så er Aalborgs største turist attraktion, den ligger ud i City Syd, det IKEA. De har 1.6 mio. årlige besøgende, tror jeg det er, så det er en meget stor turist attraktion. Det bliver selvfølgelig sagt med stor glimt i øjet og kæmpe smil, fordi de bliver ikke i traditionelt forstand opfattet som turistattraktioner, men det så sagt, samtidig med man har lidt på fornemmelsen, at der er nogen familier der bruger det som en slags søndagsunderholdning, og så tager de ud og får ”köttbullar” og cola ad libitum, og så kører de hjem igen, og så har de haft en underholdende dag der. Men ellers, hvis man skal gå tilbage til det traditionelle, så er der ingen tvivl om, at vores allerstørste lokale attraktion, det er Aalborg Zoo, som jo har tæt på en halv million gæster om året. Hvis vi kigger lidt udenfor det, så er det jo klart Faarup Sommerland, og hvis vi går lidt mere ned på det jordnære niveau, med under 100 tusinde besøgende, jamen så er det nogle af de lokale museer, Lindholm Høje muset, forsvarsmuset, kunstmuset, Utzon osv. det er de attraktioner, som skal bære udfordringen  med at skabe gæsteoplevelser på det traditionelle områder, der hedder attraktioner. \newline
\textbf{Mark:} Så Aalborg tårnet er faktisk slet ikke noget? \newline
\textbf{Lars:} Jo, jo det er det! Der er intet i vejen med Aalborg tårnet, men det eneste Aalborg tårnet som sådan kan, det er at tilbyde folk at komme op i en elevator og så nyde udsigten, ikke også? Og det er der så mange andre steder man kan. Og så kan man så få en omgang pommes frites deroppe. \newline
\textbf{Mark:} Det koster vel ikke noget at komme op, gør det? \newline
\textbf{Lars:} Jo, man betaler noget for at komme derop, med mindre du er inviteret i et eller andet selskab, så betaler man ikke noget for det. \newline
\textbf{Mikael:} Men hvad siger du så, er der andre steder man kan se, hvis nu man skal ud og se oversigten over byen, som fx heroppe, men er der andre specielle steder man måske kan se gratis, som også er sådan lidt specielle? \newline
\textbf{Lars:} Jamen hvis I gider at bruge at bruge lidt tid, så er der en masse turistinformationer på vores side, visitaalborg.dk, som i øvrigt er Aalborgs allermest besøgte site. Først og fremmest rettet mod gæsterne, men vi har også på fornemmelsen, at der er mange lokale der bruger den, vi har jo langt over 600.000 årlige unikke besøg, og man tæller først når man har navigeret på sitet i over 15 minutter, så der er rigtig rigtig meget tryk på det site. Og det kan vi jo også mærke, og det er bl.a. derfor vores partnere godt vil være med, fordi i ved, at når man har købt et basis partnerskab, så har man stort set fået sit eget profilering på vores web. Og det er stærke sager, især hvis man har noget og gå med der henvender sig til folk der er gæster i vores by, overnatninger, restauranter, attraktioner, specielle shoppingsmuligheder osv. \newline
\textbf{Mark:} Men det er lidt sjovt du nævner shoppingsmuligheder, for vi havde en diskussion i dag med vores vejleder, om hvorvidt shopping er en attraktion, for vi mente, at hvis man nu skulle fra fx Aalborg Zoo til Karolinelund, så kan man tage den hurtige rute, men det kunne være sjovt at komme forbi gågaden, hvor vi så tænkte, det er da lidt en attraktion, altså noget at opleve? \newline
\textbf{Lars:} Jamen det er jeg da helt enig med jer i. \newline
\textbf{Mikael:} Eller i hvert fald til den interessante rute.\newline
\textbf{Lars:} Men jeg tror bare, at selvom jeg er en gammel mand, så vil jeg bare sige, at man skal begynde at tænke udenfor boksen, fordi, hvad er det der gør at folk rejser tilbage til Aalborg, eller tilbage til Barcelona, eller hvor det nu er i verden, ikke også. For jeg ved udmærket godt, at når jeg tager til Barcelona, så er det først og fremmest fordi der er en utrolig behagelig temperatur dernede, når vi er på den her tid af året, ikke også, så er det typisk 10-12 grader varmere typisk i Barcelona. Og når jeg går med efterårsjakke på her, kan jeg nøjes med t-shirt ned af ”la ramblaen”. Og det andet det er så det folkeliv der er på ”ramblaen” osv, og der tror jeg, at noget af det der gør Aalborg mere attraktivt, det er sgu ikke, at de har fået en ny F16 fly, ude på museet eller en ny girafunge oppe i Zoo osv. Det er jo hele det liv som Aalborg er på vej til at udvikle, og et stykke hen af vejen, har udviklet, med de nye åbne områder på havnen, hvor man skaber atmosfære, og oplevelsesrum, hvor folk gerne vil være. Derfor synes jeg også at vores gågader de hører med til oplevelsen, jeg tror ikke, at der er nogen der rejser fra Sydnorge, ja de er så lidt specielle, lad os hold Norge udenfor, men jeg tror ikke, at der er nogen der rejser fra Tyskland, eller England eller Amsterdam, hvor vi nu også har en forbindelse til, eller fra Frankfurt, som vi snart har en forbindelse til, for at handle net undertrøjer ind i H\&M. Og jeg tror heller ikke, at de går ind for at købe en skjorte inde hos Torben Walder for så vidt, men jeg tror på, at de har lyst til at gå ud og finde nogle af de butikker, som fx kan levere noget Danish design, jeg tror på at de kan lide at gå ind i de butikker, hvor man kan levere noget unikt kunsthåndværk, ler-klaskeri, glaspusteri osv. som vi trodsalt har nogen forskellige muligheder for i Aalborg. Jeg tror de kan lide at sætte sig ned, de steder, hvor der er en behagelig fortovscafe/restaurant, for at få et lille glas og lidt at spise og nyde det liv der foregår rundt omkring, så jeg tror godt at vi kan sige, til city foreningen, for turristoplevelsens skyld, så er det ikke vigtigt, at H\&M overlever med 3 butikker i gågade systemet i Aalborg, det er vigtigt, at Inspiration og at Georg Jensen Damask og nogle af de specielle butikker, der har noget specielt og unikt, at de overlever, fordi de er med til at sætte en streg under og skabe, måske ikke skabe en ”reason to go”, men i hvert fald være med til at gøre besøget i Aalborg til noget originalt og unikt. Hvis man køber et eller andet, som man er glad for, så kan man altid huske hvor man har købt det henne, ikke? Og hvis man nu har købt det i Aalborg, så er det jo en udmærket præference at have. \newline
\textbf{Mark:} Jeg ved ikke om vi har meget mere? \newline	
\textbf{Lars:} Nej, men jeg vil gerne i relation med det her Google Maps osv. så vil jeg godt lige bede min kollega Kim om at komme ind. \newline
\textbf{Mark:} Men lige nu er vi kun i gang med analyse, vi skal først i gang med at snakke om programmering sådan lidt senere, det bliver nok om en halv måneds tid, så måske lidt senere. \newline
\textbf{Mikael:} Vi er kun i gang med at finde ud af, om det er relevant det her med ruteplanlægning. \newline
\textbf{Lars:} Årh, cool. \newline
\textbf{Mikael:} Det er egentlig bare det. \newline 
\textbf{Lars:} Men det kan jeg jo ikke svare jer på, men det kan Kim måske svare jer på, ved at sige det er er en god vej at køre ud af, eller også sige, vi har allerede ”bum bum bum”, ikke? \newline
\textbf{Mark:} Nå jo, ja, eksisterende løsninger, ikke? Hvad har i af sådan nogen? \newline
\textbf{Lars:} Ja, men skal jeg ikke bare kalde på ham alligevel? \newline
\textbf{Mikael:} Jo. \newline
\textbf{Lars:} Det er så cool, og generelt har vi det sådan i Visit Aalborg, at alle der kommer med et projekt, dem vil vi gerne give de bedste redskaber til, at gå i den mest hensigtsmæssige retning. Og grunden til, at jeg synes det var relevant, at Kim også kom ind er fordi, at han ved lidt om, hvad der er, og hvad vi drømmer om osv. Der er jeg fuldstændig blank. Især når vi kommer over i en web-baseret virkelighed. Og samtidig med det, skal det ikke være en hemmelighed, at hvis ikke vi havde en god dialog med uddannelsesstederne, så ville vores kapacitet blive reduceret med adskillge procenter. Jeg ved ikke om det er 10% eller 15%, det er måske endda 20%. Men vi har mellem fire og seks praktikanter fast baseret i huset, og vi er typisk ude efter dem, der er mindst er ude i en tre til fire måneder af gangen, for også at få noget ud af investeringen, så de kan levere noget tilbage for alt den værdigfulde information de får proppet ind i hovedet af os. Det skal man ikke undervurdere, at komme med til vores fredagsbarer. Så det synes jeg gerne vi vil bidrage til. Nu synes jeg i skal til at fortælle lidt om hvor i er henne, og det her er meget gryn osv, så take it away boys. \newline
\textbf{Mark:} Vi læser Software ude på universitetet på Strandvejen, og vi skal skrive projekt om ruteplanlægning, hvor vores initierende problem er, at GoogleMaps kun kan vise en rute fra et sted til et andet. \newline
\textbf{Kim:} I vil gerne lave en form for flerpunktsrute? \newline
\textbf{Mark:} Ja præcis, og hvordan man finder den bedste. Så har vi turisme som case, fordi det var interessant, hvad turisterne havde lyst til, og hvis de havde en masse seværdigheder de gerne ville se, hvordan kommer de så fra det ene sted til det andet. Og om de bare har lyst til at komme fra det ene sted til det andet, eller om de også har lyst til, hvis nu programmet kan foreslå en mere interessant rute, som andre har set eller rated højt. Lige nu er vi kun igang med at høre om turisme i Aalborg, fordi vi ikke er så langt med projektet endnu. \newline
\textbf{Kim:} Men i stedet for at folk kommer til at tænkte ud af et spor, og tænker ”det har vi allerede lavet”, eller et eller andet, så er det måske relevant lige at spore den en lille smule. \newline
\textbf{Mikael:} Indtil videre har vi fundet TripAdvisor, som kan finde frem til et sted. Det er den eksisterende løsning der ligger tættest op af det, vi har tænkt os at lave. \newline
\textbf{Mark:} Det var mere om i havde et eller andet? \newline
\textbf{Kim:} Nej, vi har ikke noget ala det faktisk, så det lyder da rigtig spændende. Det lyder som en god idé i hvert fald. Det lyder interessant for både turisten og os. Så det ville blive rigtig godt at pushe videre til brugerne. \newline
\textbf{Mark:} Der er ikke så meget i det, da vi ikke er så langt, kan man sige. \newline
\textbf{Kim:} Vi kan da helt sikkert hjælpe jer med, at tilrettelægge lidt om, hvad er vores erfaring med, hvad turisterne gerne vil have. Tendensen med sådanne programmer er tit, synes jeg, at man prøver at lave et program der kan helt vild meget, og så ender en med at være værdiløs, da den kan alt for meget, og ikke har noget fokus. Så det der med at snævre det lidt ned og sige, ”vores program, den kan lige præcis hjælpe dig med at finde den her tur”. Det er det, der får en til at vælge at downloade den. \newline
\textbf{Mark:} Vi har fået at vide, at vi kun må skrive i programmeringssproget ”C”, så vi må egentligt kun lave et lille program til en computer. \newline
\textbf{Lars:} Der skal jo mange sten til at bygge et hus, så hvis man skaber et stærkt fundament, så kan man bygge videre på det på et tidspunkt. Så er der måske bedre plads i forløbet til det. \newline
\textbf{Mikael:} Så har man måske også nogle bedre redskaber. \newline
\textbf{Kim:} Det er vel en art af forundersøgelse af et man kan tage fat på senere hen, men vi vil da helt sikkert gerne hjælpe. \newline
\textbf{Lars:} Men gutter, nu har i jo også Kim direkte at referere til. Det skal retfærdigvis siges til vores og til Kims og hele husets undskyldning, at lige nøjagtig vore afdeling er lidt hårdt spændt for i øjeblikket. Vi har vores web. Jeg så i øvrigt at i fredagsmailen, at vores webmaster er gået på barsel, så nu laver vi en side til nyfødte forældre. Nu er han jo på barsel, og vi har en praktikant deroppe. \newline
\textbf{Kim:} Vi har tit noget i gang, og vi vil helt vild gerne. Fordi vi også får en masse ud af det, vi får noget viden og nogle skide gode idéer fra jer, og nogle gange kører det også videre, de projekter i nu har gang i. Så vi er altid klar på en indledende snak og hjælp med lidt retning på projektet. \newline
\textbf{Lars:} Kim, tak for det. Nu vil jeg slutte af med en lille snak. Det hører også med til historien, at vi har i mange år udgivet forskellige publikationer, igen for at gøre hvad vi kan, for at skabe den bedst mulige gæste-oplevelse. Det har vi selvfølgeligt dels gjort i ”værgefasen”, altså den fase når vi foreksempel tager til syd-norge eller nord-tyskland, eller til england eller whatever, hvor vi prøver at værge gæster til vores by. Men også i hele gennemførelses-fasen, i den tid hvor folk er kommet, så sikrer at de får en god oplevelse. Så har vi blandt andet haft det her by-kort, Aalborgs eneste autoriseret by-kort, der er flere der har prøvet på at planke os, og lave noget tilsvarende. Men de når os ikke til sokke-holderne, for det første fordi, at vi er det eneste kort, der har turist informationer, og værdifulde oplysninger, og for det andet er vi også de eneste udgivere som regelmæssigt distribuerer kortet ud til nøglesteder, hvor gæsterne er. Hoteller, restauranter, butikker, overnatningssteder, alle de steder hvor turisterne kommer fra, står de her steder før man kommer til Aalborg. Så det er delt sådan op, at vi har det store by-kort, indklusiv City-Syd osv. på den ene side, og Nørresundby, som er rigtig vigtig, på den anden side. Og på modsatte side af kortet, der har vi så det, som vi må sige er den mest anvendte, er vores indre by-kort. Her refererer vi til alle mulige forskellige institutioner, som har overnatnings ting osv. For godt to år siden, har vi skåret alle vores printmedier fuldstændig væk, fordi  der synes vi, at vi var blevet elektroniske, og nu skulle alt ske på QR-kode basis osv. Men vi måtte erkende at vi havde gjort regning uden værk, det var for tidligt, folk har stadigvæk brug for, og vil gerne have, et stykke papir i hånden. Men så gik vi lidt i tænkeboks, og har nu sidenhen lavet nogle nye publikationer, som vi gjort første gang i år, hvor vi lavede en lille Quick-Guide. Man kan ikke annoncere i den, men man kan købe en optagelse, hvor man har en halv side. Det er så kun for virksomheder inden for shopping, dining og attraktioner. Altså restauranter, caféer og så attraktioner, det er kun dem vi vil have her. Fordi vi gerne vil have, at det skal være en slags guide, som folk kan bruge når de er kommet til byen og går rundt. Den bliver distribueret i 25.000 eksemplarer, og den har vi lavet 250.000 eksemplarer af. Altså ¼ millon eksemplarer, den gælder så i et helt år. Og 20.000 af de 250.000 bliver lavet på engelsk. Den anden her er så lavet sådan, at der inde i midten er et lille mini-udprint af vores by-kort, så man kan finde rundt i byen, fordi det er formålet – Der er ikke nogen grund til at lede folk ud i City-Syd, da de fleste vil være inde i byen. Så er der lidt praktisk information, attraktioner, Zoo, Utzon centeret, Kunsten, Lindholm Høje, osv. Og så der er noget café-halløj, restauranter, og sidst men ikke mindst nogle attraktioner, som shopping-muligheder, som har valgt at købe en optagelse. Den gentager vi så i 2015, plus at vi laver, målrettet mod sommergæsterne, et magazin i A4-format, som også kommer i 25.000 eksemplarer, som først og fremmest skal lave lidt redaktionelt samt noget annoncering, teasing for, hvad man kan lave hen over sommeren i Aalborg. Hvor vi blant andet i år havde noget Tall-Ships-Race, og Blå Festival, og flere forskellige andre større projekter på Beddingen. Og som noget andet nyt, laver vi også dette magazin format som en cruise-udgave, som er målrettet mod de gæster der kommer fra sø-siden, vi er jo i den heldige situation, at vi skal være værter for 15 stor-krydstogtsskibe, og næsten 15.000 gæster henover sommeren 2015. Vi tilbyder gæterne en unik oplevelse, og lægger ekstrem meget vægt på værtsskab, og især det gode værtsskab. Måden at gøre det på, er ved at give dem oplevelser som er unikke, og som rammer dem i hjertet på en eller anden måde, og der valgte vi for sjov skyld, for et par år siden, at i stedet for at have et orkester som stod og spillede musik, så valgte vi at lave en give-away, så de fik en rigtig dansk hotdog, serveret fra en pølsevogn, og det synes folk var helt skørt. Men de elsker jo smagen, og de synes jo det er fantatisk. Det kan de jo ikke få nogen som helst andre steder på deres cruise. Det blev en rigtig populær begivenhed. Men gutter, jeg synes i skal gå tilbage til jeres studie-kammer, og så håber jeg at det har bragt lidt info osv. 
