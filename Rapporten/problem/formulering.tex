\chapter{Problembeskrivelsen}

I dette afsnit vil den endelige problemformulering blive beskrevet, samt kravene til den endelige løsning. Til sidst afgrænses der i forhold til gruppens evner og de krav der er blevet opstillet. 


\section{Afgrænsning}
\textbf{Skrives når problemanalysen er færdig} \newline
Spørgeskemaet bekræftede hypotesen, om at turister har problemer med at finde rundt i storbyer. Dermed blev det bekræftet, at en løsning der kan hjælpe turister med at finde rundt, kan være aktuel. Gennem spørgeskemaet, blev det også klart, at turister helst vil have en interessant rute og opleve byen, i stedet for at komme så hurtigt gennem deres på forhånd valgte attraktioner, som muligt. 


\section{Problemformuleringen}
Hvordan udvikles der en softwareløsning, der hjælper turisten med at finde rundt i en storby, på en interessant rute mellem turistens egne valgte attraktioner?

\section{Kravspecifikationer - optimale løsning}

\begin{itemize}
	\item Programmet skal udvikles som en applikation til moderne smartphones, inklusiv iOS, Android og Windows Phone. 
	\item Programmet skal kunne beregne den korteste rute mellem en række punkter.
	\item Det skal være i stand til at give forslag til en anden rute, der inkludere attraktioner der ligger tæt på ruten.
	\item Programmet skal vise et kort med ruten, og på kortet vise attraktioner der er tæt på. 
	\item Det skal være muligt at få programmet til at vise rutevejledningen, på samme måde som normale GPS-enheder. 
	\item Der skal være mulighed for at downloade en offline version af den rute der er valgt, inklusiv kortet for det omkringliggende område. Dette er for at hjælpe dem der ikke har billig data hvor de bruger programmet.
	\item Programmet skal give mulighed for at bedømme attraktioner, og derved tildele dem en rating. Dette skal bruges til at rangere attraktionerne, for at hjælpe turisterne med at vælge den mest interessante rute. 
\end{itemize}

\section{Kravsspecifikationer}
Gruppen har gennem spørgskema og interview, fået stillet en række krav til løsningen, af turister og VisitAalborg. 
Gennem spørgskemaet, blev det konkluderet, at det vigtigste for turister, er at de kan opleve byen på en interessant rute. 
Derudover har turistbureauet givet udtryk for, at løsningen gerne skal være så simpel som mulig, og løse en specifik opgave godt, da de mener, at det er i turistens bedste interesse. \newline
Der er blevet stillet krav fra universitets side, om at programmet skal være et lille specifikt program i C, af høj kvalitet. Dette stemmer godt overens, med de krav der er blevet stillet fra turistbureauets side.   \newline

Dette afsnit vil opsætte kravene for programmet og dets funktionaliteter, og vil ligge til grundlag for udviklingen af programmet. 

\begin{itemize}
	\item Programmet skal kunne beregne den korteste rute mellem en række punkter.
	\item Det skal være i stand til at give forslag til en anden rute, der inkludere attraktioner der ligger tæt på ruten.
 	\item Der skal som output, gives en liste over rutens destiationer, sorteret efter hvilken rækkefølge er hurtigst.
 	\item Programmet skal som minimum være kompatibel med Windows styresystemet. 
\end{itemize}

Da dette er et P1 projekt, og gruppen er begrænset af både tid og erfaring, har gruppen valgte at begrænse softwareløsningen, på følgende punkter \newline
\begin{itemize}
	\item Rutevejledningen bliver i fugleflugtslinje.
	\item Brugeren kan kun vælge destinationer ud fra en række forudbestemte punkter.
	\item Tekstbaseret brugergrænseflade.
\end{itemize}

\section{Løsningsforslag}
På baggrund af kravene og afgrænsningen, har gruppen tænkt sig at lave et program, som har nogle forudbestemte destinationer, der dækker over destinationerne i Aalborg, hvorefter brugeren vælger de destinationer han/hun ønsker at besøge. Programmet vil ud fra disse punkter, beregne den korteste rute, og undersøge om der er andre attraktioner, som ligger tæt på ruten, og spørge brugeren, om det kunne være interessant at besøge disse steder. Hvis ja, vil disse punkter også blive inkluderet. Resultatet bliver en liste over destinationerne, der står i rækkefølge, så turisten ved hvilken rækkefølge de skal besøge dem i, for at få den mest optimale rute.
