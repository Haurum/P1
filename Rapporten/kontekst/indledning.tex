\chapter{Indledning}

Hvert år besøger flere millioner turister Danmark, hvilket er godt for den danske økonomi. Når turisterne bruger penge på en dansk vare, service eller oplevelse, bliver det sådan set eksporteret til udlandet – og derfor bliver dette betragtet som en eksportvare. I alt står denne eksport type for 3,6\% af den danske eksport. Turisterne har et forbrug på 87,2 mia. kr., hvoraf de udenlandske turister bruger 35,7 mia. kr. altså godt 41\%, mens de danske turister står for de resterende 59\%. Udover at turismen hjælper det danske samfund økonomisk, skaber turismen ifølge VisitDanmark knap 122.500 fuldtidsjobs. \citep{faktaogtalVD}  \newline
Det ses gerne at turisterne kommer tilbage til Danmark igen. Dette sker naturligvis ved, at turisterne nyder deres ophold og får den bedst mulige ferie. Som turist i en storby kan det forekomme svært at finde rundt, og samtidigt virke let at fare vildt. Hvis en turist i København gerne vil se Rundetårn, skal turisten blot kigge efter det store monument, og gå i den retning hvor attraktionen nu er. Dog kan det ske at turisten undervejs mister tårnet af syne, og pludseligt ved turisten ikke i hvilken retning personen nu skal gå. Turisten kan vælge at bruge sin smartphone, hvis turisten da er i besiddelse af en, og kan eksempelvis gå på internetsiden GoogleMaps. Her kan turisten så finde en rutevejledning fra punkt A til B, dog vil der kunne opleves problematikker, hvis en flerpunkts rute ønskes. Idette projekt har gruppen valgt at afgrænse sig til at kigge på en enkelt dansk by, i dette tilfælde valgte gruppen Aalborg, da dette var mest oplagt.  \newline

For turisten vil planlægning på forhånd være en god ting, hvis turisterne vil nå så mange attraktioner som muligt på en ferie, da tiden kan være begrænset\textbf{INDSÆT KILDE HER!!}. Hvis feriedestinationen er i udlandet, ville en offline løsning til ruteplanlægning være optimal, da brugen af mobildata i udlandet kan koste mange penge\citep {TDC}. \newline
Samtidigt kan der spørges, hvad der gør en rute god: Er det hvor hurtigt turisten kommer fra den ene valgte attraktion til den anden? Kan der findes en mere interessant rute, eventuelt med attraktioner der ikke er oplyst i rejsebureauets brochurer, måske en smutvej forbi havnen eller muligheden for en flerpunkts rute mellem attraktionerne? Der kan være mange parametre der spiller ind, når den foretrukne rute skal vælges. \newline

I dette projekt har gruppen valgt at kigge på: I hvilket omfang ruteplanlægning kan hjælpe turister med, at finde den hurtigste eller mest interessante rute mellem attraktioner i Aalborg? Herunder kan der kigges på hvad en interessant rute ville være, hvilken form for rute turisterne foretrækker og i hvilke situationer har turister brug for ruteplanlægning og hvorfor? Derudover vil gruppen kigge på hvilke typer af turister der er relevante i forhold til dette projekt og hvad deres behov kunne være. \newline \newpage

\subsection{Begreber}
Ved en ’interessant rute’ vil der i denne rapport refereres til en rute, der indkluderer unikke kulturelle og nationale oplevelser, som dog er individuel for turisten. I appendix (interview) sagde Lars fra VisitAalborg, at interessante attraktioner i aalborg blandt andet er Aalborg Zoo, Utzon centeret, kuns1tmuseet, gågaden, havnen og City Syd. En rute der indkluderer danske traditioner og spise.\newline

En ’attraktion’ vil i denne rapport forståes som både traditionelle, og utraditionelle attraktioner. Traditionelle attraktioner i Aalborg er Aalborg Zoo, kunstmuseet og lignende. De utraditionelle indkluderer steder som City Syd, Ikea og Jensens Bøfhus.\newline
’Turist’ vil i denne rapport være til dels defineret ud fra UNWTOs (UN World Tourism Organistation) beskrivelse, som beskriver to slags turister. En-dagsturister, som maksimalt overnatter én nat på stedet, og en generel turist, som overnatter flere end en nat. Udover disse to typer ser vi også en undergruppe ’erhvervsturist’. Dette er turister, der kommer til stedet med et arbejdsformål. \newline
En turist er en som har et bestemt formål med rejsen, til forskel for besøgende og rejsende. Formålet kan fx være, at opleve den danske kultur. \newline
En ’storby’ i Danmark, er i denne rapport en by, omgivet af områder med lavere bebyggelse.

