\section{Spørgeskema}
I dette afsnit vil man kunne se udformningen af spørgeskemaet og de resultater som der vil kan uddrages fra spørgeskemaet. I interessesentanalysen blev turister placeret som en delvis resourceperson, og ud fra dette har gruppen valgt at lave et spørgeskema, som skal hjælpe med at besvare den initierende problemstilling. Dette blev gjort med den hensigt at få så mange synspunkter på de opstillede problemstillinger, så det ville være muligt at generalisere ud fra besvarelserne. Spørgeskemaet bliver udarbejdet vha. den kvantitative metode, i form af et internetinterview. Se i apendix A1 og A2 om mere information om den brugte teori og rådata.

\subsection{Udformning}
Først udformede vi en række problemstillinger vi ville have svar på i vores undersøgelse: 
\begin{itemize}
\item Hvilke slags attraktioner tager turister til storbyerne for at se?
\item Hvordan planlægger turister deres ferie?
\item Hvordan finder turister rundt?
\item Har turister problemer med at finde rundt?
\item Fortrækker turister den hurtigste eller den mest interessante rute?
\item Er den løsning vi har i tankerne noget respondenterne ville bruge?
\end{itemize}
Da gruppen havde valgt et internetinterview i form af en online undersøgelse, valgte gruppen at dele spørgsmålene på Facebook. Dette vil dog give nogle begrænsninger: Spørgeskemaet kan kun ses af personer som gruppen er venner med på Facebook, hvilket mest vil være unge mennesker. Derudover er spørgeskemaet lavet på dansk, for at gøre spørgsmålene så let forståelige som muligt for gruppens venner på Facebook. Da mange af de personer gruppen kender på Facebook sikkert ikke har været turister i Aalborg, blev spørgsmålene rettet mod alle storbyer.
\subsection{Resultatbehandling}
Spørgeskemaet var lagt op på Facebook i tre dage, hvorefter resultaterne blev behandlet. I alt var der kommet 60 besvarelser, hvilket giver gruppen en relativ lille respondentgruppe at arbejde med, dog giver resultaterne nogle klare tendenser, som gruppen har arbejdet ud fra. Herunder kan der ses forklaring på formålet med spørgsmålet, og hvad besvarelserne kan fortælle.

\textbf{Hvad er vigtigt for dig på din storbyferie?}\newline
Formålet med dette spørgsmål er at finde ud af hvad turister gerne vil se eller opleve i en storby, for at kunne se hvad der eventuelt kunne implementeres i projektets løsning.\newline
Ud fra besvarelserne afgivet af respondent gruppen (som kan ses i apendix A2), kan der ses hvad respondenterne vægter højest på deres storbyferie: Se byens seværdigheder, opleve kulturen, maden og shopping. Derudover var der en del der havde kommentereret, at de kom til storbyen for at se sportsbegivenheder. Dette var dog ikke en valgmulighed på spørgeskemaet, så det kan ikke uddrages til projektet, da der kan have været en mulighed for at respondenterne ikke havde tænkt over dette svar.\newline

\textbf{Hvilke hjælpemidler bruger du til at planlægge din storbysferie?}\newline
Formålet med dette spørgsmål er, at finde eksisterende planlægningsværktøjer, hvor der kunne kigges på fordele og ulemper til dette projekt.\newline 
I besvarelserne kan der ses en liste af eksisterende planlægningsværktøjer, hvor de både er elektroniske og i form af brochurer og lignende.\newline
 
\textbf{Hvilke redskaber bruger du til at finde rundt når du er på storbyferie?}\newline
Formålet med dette spørgsmål er at finde ud af om turister brugte redskaber til at finde rundt.\newline 
Ud fra besvarelserne på dette spørgsmål kan der ses, hvilke redskaber respondentgruppen bruger, til at finde rundt på ferier. Her har respondentgruppen mulighed for at svarer på mere end én ting. Der kan ses at 81,67\% bruger diverse kort og brochurer, og 55\% bruger elektroniske redskaber til at finde rundt. \newline

\textbf{Har du nogensinde haft problemer med at finde vej på din storbyferie?}\newline
Formålet med dette spørgsmål er at få afklaret om et af de problemer vi opbygger vores projekt omkring, faktisk er et problem.\newline
Ud fra besvarelserne kan der ses, at hele 68,33\% altså hele 2/3, har haft problemer med at finde rundt på deres storbyferie.\newline

\textbf{Når du skal fra en aktivitet til en anden på din storbyferie, vil du helst tage den hurtigste rute eller en langsommere men mere interessant rute?}\newline 
Formålet med dette spørgsmål er, at finde ud af hvad turisterne helst ville have, en hurtig rute fra A til B, eller en interessantrute, som er langsommere, hvor man får set andre ting på vejen  til sin destination.\newline
Der ses tydeligt at vores respondenter fortrækker den interessante rute over den hurtigste rute, med henholdsvis 80\% for den interessante og 20\% for den hurtigste. Folk er mere interesseret i at få flere oplevelser, end at komme hurtigt frem til næste punkt på dagsordenen.\newline

\textbf{Et program/applikation, som hjælper mig med at finde den hurtigste og/eller mest interessante vej igennem byen, via mine valgte ”must see” destinationer, ville være noget jeg kunne bruge?}\newline
Formålet med dette spørgsmål er at finde ud af om projektet egentlig har nogen interesse hos brugeren.\newline  
I dette spørgsmål svarede 90\% at den ideelle løsning ville enten kunne bruges eller ønskes.