\chapter{Indledning}

Hvert år besøger flere millioner turister Danmark, hvilket er godt for den danske økonomi. Når turisterne bruger penge på en dansk vare, service eller oplevelse, bliver det sådan set eksporteret til udlandet – Og derfor bliver dette betragtet som en eksport vare. I alt står denne eksport type for 3,6\% af den danske eksport. Turisterne har et forbrug på 87,2 mia. kr., de udenlandske turister bruger 35,7 mia. kr. altså godt 41\%, mens den danske befolkning står for de resterende 59\%. Udover at turismen hjælper det danske samfund økonomisk, skaber turismen ifølge VisitDenmark knap 122.500 fuldtidsjobs. \citep{faktaogtalVD}   \newline
Det ses gerne at turisterne kommer tilbage til Danmark igen. Dette sker naturligvis ved at turisterne nyder deres ophold og får den bedst mulige ferie. Som turist i en storby, kan det forekomme svært at finde rundt, og kan let fare vildt. Hvis en turist i Paris gerne vil se Eiffeltårnet, skal turisten blot kigge efter det store monument, og gå i den retning hvor attraktionen nu er. Dog kan det ske at turisten undervejs mister Eiffeltårnet af syne, og pludseligt ved turisten ikke i hvilken retning personen nu skal gå. Turisten kan vælge at bruge sin smartphone, hvis turisten da er i besiddelse af en, og kan eksempelvis gå på internetsiden GoogleMaps. Her kan turisten så finde en rutevejledning fra punkt A til B, dog vil der kunne opleves problematikker, hvis en flerpunkts rute ønskes. Dette behøver ikke kun at ske i storbyer så som Paris, dette kunne også ske i nogle af de større danske byer. \newline
Turister vil generelt opdage, at planlægning på forhånd er en god ting, hvis turisterne vil nå så mange attraktioner som muligt på en ferie, da tiden kan være begrænset. Dog kan der samtidigt spørges, hvad der gør en rute god: Er det hvor hurtigt turisten kommer fra den ene valgte attraktion til den anden? Kan der findes en mere interessant rute, eventuelt med attraktioner der ikke er oplyst i rejsebureauets brochurer, måske en smutvej forbi havnen eller muligheden for en flerpunkts rute mellem attraktionerne? Der kan være mange parametre der spilder ind, når man skal vælge den foretrukne rute. \newline
Hvis en rejse skulle gøres mere interessant for turister, kan der så udvikles et program, der hjælper turisten med at finde den foretrukne rute mellem attraktionerne? Hvad vil være den foretrukne rute: Den hurtigste, eller den mest interessante? Og i hvilket omfang vil dette program kunne hjælpe turisten? \newline
At lave et program der gælder for hele verdens storbyer, ville kræve meget mere programmerings erfaring og tid. Derfor har gruppen i dette projekt valgt at afgrænse sig til at arbejde med ruteplanlægning for turister i Aalborg. Der kan også være forskellige opfattelser for hvad hver person ser som en attraktion. Gruppen har derfor valgt at give nogle eksempler på hvad gruppen mener er attraktioner i Aalborg. Det kunne bl.a. være kulinariske oplevelser på diverse restauranter. Kulturelle attraktioner som Tyren, museer, havnen, koncerter og andre arrangementer. Det kunne være oplevelser som en tur i Zoo eller Aalborgtårnet. Steder som city syd, gågaden og Jomfru Ane Gade mener gruppen også kan gå ind under attraktioner.
