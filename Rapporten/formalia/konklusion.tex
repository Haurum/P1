\chapter{Konklusion \& Perspektivering}
I dette kapitel vil der blive konkluderet på, om løsningen besvarer den opstillede problemformulering fra problembeskrivelsen.

Problemformulering:\newline
\textit{"Hvordan udvikles der en softwareløsning, der hjælper turisten med at finde rundt i en storby, på en interessant rute mellem turistens egne valgte attraktioner?"}

Igennem processen af problemløsningen, blev der udviklet et stykke software, der hjælper turisten med at finde en interessant rute, i fugleflugtslinje. Denne fugleflugtslinje er der ikke et kort over, hvilket betyder, at turisten ikke bliver hjulpet i et særlig stort omfang. Hjælpen fra denne softwareløsning er blot et forslag for, hvilke attraktioner der skal besøges, og i hvilken rækkefølge med kortest mulig distance mellem disse. Den interessante rute udgøres af de attraktioner brugeren vælger, samt de attraktioner der er mulighed for at tilføje, når de bliver spurgt om yderligere attraktioner til deres rute, hvis der findes attraktioner tæt på deres nuværende rute.\newline
Et optimalt hjælpemiddel til at finde rundt i en storby ville være en løsning der kortlægger ruten, hvilket dette program ikke har formået. En eventuel løsning på dette, ville være at sætte attraktionerne ind på et kort, ved hjælp af de allerede benyttede koordinater. Problemet er at finde et kort der fungere i C, og ikke fx JavaScript som Google Maps benytter.

Der kan tilnærmelsesvis siges, at det software der i denne rapport er udviklet stadig hjælper turisten med at finde rundt, da den giver en liste med rækkefølgen over attraktionerne, som brugeren har bestemt. Denne løsning kan ikke vise vej uden at brugeren selv indtaster attraktionerne i fx Google Maps. Programmet kan hverken kortlægge eller guide, men fortæller udelukkende om rutens planlagte rækkefølge.

\textbf{Perspektivering} \newline
En fremmed storby kan være svær at finde rundt i, og det kan være svært at finde ud af hvilken rute er bedst at tage for at få set så meget som muligt, uden at spilde tiden med at gå rundt og fare vildt. En softwareløsning kan være løsning på problemet, men den skal være smart, og fungere på mobile enheder, så brugeren altid kan trække den frem af lommen og få rutevejledning med det samme. Det kunne være det næste skridt for den softwareløsning der af gruppen er blevet udviklet, i hvert fald når den kan tage højde for veje, og ikke længere kun fungere i fugleflugt. Den ideelle løsning, der tidligere i rapporten er beskrevet, ville være der hvor projektet skulle ledes hen, hvis der skulle udvikles videre på projektet. En smartphone app, der også fungere offline, er en god løsning, der både kan hjælpe udenlandske turister uden internetforbindelse, og turister fra samme land. \newline
Problemet med at finde rundt på en flerpunktsrute er ikke begrænset til turister, der gerne vil se så mange attraktioner som muligt på en dag, men er også et problem hos fx hjemmeplejen, varelevering, postbude eller lignende. Gruppens softwareløsning er derfor ikke begrænset til at løse et problem for turister, men kan i ligeså stor udstrækning løse det samme problem for andre brancher eller personer, selvfølgelig uden den del med den interessante rute. Forskellen er behovet for at få så præcis en rute som muligt, og hvor lang tid der kan ofres på at beregne ruten. Nearest neighbour algoritmen fungere godt i tilfældet med turister, fordi det er en algoritme der kan beregne mange punkter hurtigt, men desværre også har sine usikkerheder, fordi den ikke nødvendigvis finder den korteste rute, men kun tilnærmelsesvist. Dette betyder ikke så meget for en turist der bare skal gå et par kilometer, men kan hurtigt komme til at betyde meget for firmaer der kører flere hundrede kilometer dagligt, og andre algoritmer vil derfor være nødvendige i andre sammenhæng. 
