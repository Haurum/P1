\section{Teknologianalyse} \label{sec:teknologianalyse}


\subsection{GPS}

I 20200 kilometers højde over jordens overflade, er der som minimum 24 satellitter i kredsløb hele tiden, der hver tager en tur rundt om jorden, to gange i løbet af et døgn. Disse satellitter udgør hvad der bliver kaldt ”The GPS Constellation”, eller på dansk, GPS stjernebilledet.  GPS-systemet bruges til at bestemme positioner på jorden, og vedligeholdes af den amerikanske stat, der sørger for at nye satellitter bliver sendt op for at erstatte de gamle, og for at de satellitter der er i kredsløb nu, bliver vedligeholdt og holder deres kurs. I øjeblikket har den amerikanske stat, 31 satellitter i kredsløb, hvoraf de ældste er af typen IIA, der blev sendt op mellem 1990-1997. Hver generation af satellitter indeholder nyere teknologi, mere præcise ure og større præcision, for hele tiden at forbedre systemet, både til gavn for civile/private brugere af GPS systemet, men også til nytte for militæret, som systemet oprindeligt blev udviklet til. 
I satellitten er der indbygget atomure, der med stor præcision bestemmer hvor lang tid der er gået siden d. 5/1-1980, den såkaldte GPS tid. Denne information sender satellitten, sammen med informationer om satellittens position, via radiobølger, der bevæger sig med lysets hastighed mod jorden. På jorden modtager GPS-modtagere disse radiobølger, og notere det præcise tidspunkt for modtagelsen af signalet. Baseret på denne information, kan modtageren bestemme distancen mellem den selv og satellitten, ved at multiplicere hastigheden af signalet (lysets hastighed i m/s), med den tid det tog signalet at bevæge sig fra satellitten, altså tidspunktet for modtagelse af signalet, minus den tid satellitten noterede at den sendte signalet. Når en GPS-modtager får signal fra 3 forskellige satellitter, kan den bestemme dens aktuelle position i to dimensioner, og med hjælp af en fjerde satellit, kan højden også bestemmes.
I de fleste moderne smartphones, sidder der i dag en lille GPS-modtager, så brugeren til enhver tid kan finde ud af hvor vedkommende er, og bruge det til at finde en rute fra et punkt til et andet. Problemet med GPS teknologien er dog, at det i gennemsnit tager mellem 30 og 40 sekunder at få en position, da den udelukkende er afhængig af at fange radiosignaler fra satellitterne, der bevæger sig rundt i en højde af cirka 20200 km.  En nyere teknologi kaldet A-GPS (Assisted GPS) er derfor blevet udviklet, og er nu blevet standard i mobiltelefoner. A-GPS teknologien, bygger på mobiltelefoners opkobling til telefonmaster, for hurtigere at finde positionen. Telefonmasterne har typisk selv en GPS-modtager indbygget, og modtager derfor hele tiden informationer fra GPS’er, og kan derved være med til at hjælpe modtageren i mobilen med at tyde signaler fra satellitterne, der ikke er fuldkomne. GPS-modtageren i en mobiltelefon kan også få serveret informationer om hvor satellitterne befinder sig i forhold til den selv, da masterne typisk kender mobilens omtrentlige position, og om ting i vejret eller atmosfæren der kan have en indflydelse på modtagelse af signaler fra satellitterne. 
Tre telefonmaster kan fungere som et simpelt positionerings system, der basalt set fungere ligesom satellitterne, og det er alt den ekstra information modtagerne får, der gør at A-GPS typisk er både hurtigere og mere præcis, end almindelig GPS.

\section{Eksisterende løsninger}
\subsection{FindTheBestRoute.com}
Google Maps er begrænset til kun at kunne vise vejen fra et punkt til et andet. Det har FindTheBestRoute.com taget kampen op imod, og har derfor lavet en hjemmeside på www.findthebestroute.com, hvor den hurtigste rute mellem 10 forskellige punkter kan beregnes. FindTheBestRoute.com, udnytter Google Maps JavaScript API v3, altså en grænseflade til Google Maps, der tillader andre programmer at benytte Google Maps, til fx at få vist et kort, eller beregne en rute \citep{ftbr}. Selvom der på maps.google.dk ikke er mulighed for at indtaste forskellige destinationer, og få anvist den hurtigste rute imellem punkterne, så har Google Maps faktisk allerede funktionaliteten indbygget til at foretage denne beregning, baseret på ”the Travelling Salesman Problem”. For findthebestroute.com, er det derfor simpelt at sende en anmodning til Google, der indeholder informationer om de forskellige destinationer der skal forbindes med en rute. Google foretager så beregningerne, og sender den bedste rute tilbage til findthebestrute.com, hvor de så kan vise ruten til deres brugere \citep{googleapi}.

\subsection{TripAdvisor Offline City Guides}
TripAdvisor har lavet en offline app, som kan hjælpe med at guide turister rundt, i den by de er rejst til. Den har mange forskellige funktioner, som fx et kort indlagt i appen. Dette kort er rigtig effektivt hvis brugeren har forberedt sig hjemmefra, fordi turisten kan downloade et kort over den by brugeren skal til, så den kan fungere offline. Det er en stor hjælp for turister, da ingen gider bruge en masse penge på mobildata. \newline
Appen fungere sådan, at der ligger information omkring den by brugeren rejser til inde på appen, som brugeren kan downloade, så disse informationer er tilgængelige offline. Så når turisten er taget på ferie og mangler hjælp til, hvad byen har at tilbyde, kan turisten gå ind og tjekke appens ideer og forslag. Her er der kategorier som restauranter, hoteller, attraktioner, byliv og shopping. Indenfor hver kategori er det så muligt at vælge “Best in Town”, og så komme ind på en topliste over fx attraktioner i den by der er valgt.\newline
Hvis brugeren klikker videre på fx en attraktion, kommer der informationer og funktioner. Her kommer der informationer omkring attraktionen, hvordan stedet/oplevelsen har været for andre brugere af appen. Her kan de så give den point fra 1-5, og kan skrive kommentarer til stedet. Hvis stedet så er noget for brugeren, er der en knap, der vil vise hvor i byen stedet ligger, men der er også en knap, der vil vise hen til stedet, så brugeren ikke selv skal finde frem via kort.
Denne app har rigtig mange gode funktioner, en af de rigtig gode er det offline kort. Dette gør det muligt at undgå at bruge mobildata, på en udlandsrejse, og gør det muligt hele tiden at have et kort ved hånden. Ved siden af det, kan der fås et indblik i, hvilke ting der er at se og opleve, i den valgte by, med kommentarer og ratings fra andre brugere, der har besøgt disse steder.
Det appen kunne mangle var en mulighed for, at vælge flere seværdigheder på deres liste, og give en rute mellem disse seværdigheder, så det er muligt at få en flerpunkts rute, så brugeren fx kunne gå ind på toplisten over attraktioner, og krydse af i top 3, og så få den hurtigste rute mellem disse 3 attraktioner. 

