\section{Eksisterende løsninger}
I spørgeskemaet blev der opremset en hel række af hjælpemidler som respondenterne bruger på deres storbyferie, hvilket til projektet vil være eksisterende løsninger. I det kommende afsnit vil der blive kigget på TripAdvisors app, hvilken var en af de hjælpemidler som respondent gruppen havde nævnt. Der udover har gruppen fundet Fidnthebestrute.com, som vha. Google Maps kan lave en flerpunktsrute.  

\subsection{TripAdvisor Offline City Guides}
TripAdvisor har en offline app, der kan hjælpe med at guide turister rundt, i den by de er rejst til. Den har mange forskellige funktioner, som fx et kort indlagt i appen. Dette kort kan være effektivt, hvis brugeren har forberedt sig hjemmefra. Dette skyldes, at turisten kan downloade et kort over den by, brugeren skal besøge, og derved vil den fungere offline. Grunden til at det er effektivt for turister, kan skyldes at mobildata kan være dyrt i udlandet\citep {TDC}. \newline
Udover et kort, har app'en også nogle informationer omkring de mange forskellige byer. Disse informationer har turisten ligeledes mulighed for at downloade, så de også er tilgængelige offline. Ved hjælp af disse informationer, kan turisten fremskaffe sig hjælp, hvis turisten fx er interesseret i at finde en restaurant, finde et hotel, se en bestemt attraktion eller lignende. Ønsker brugeren at besøge en af attraktion eller lignende, kan der ved hjælp af en knap, klikkes frem til en lokalisation, som turisten enten selv kan finde vej til, eller benytte en anden knap i app'en og få indlagt ruten i det downloadede kort.\newline
Hvert af disse kategorier indeholder en ”Best in Town”-funktion, som er en liste over de mest populære attraktioner, ifølge TripAdvisors brugere af app'en, da der er et point-system, som giver brugere af app'en mulighed for at vurdere og skrive kommentar til de enkelte attraktioner, i en skala på 1-5. \newline

TripAdvisors app har mange gode funktioner. En af de gode funktioner, er det offline kort, der giver mulighed for at undgå brugen af mobildata, på en udlandsrejse, og gør det muligt hele tiden at have et kort ved hånden. Herudover kan der fås et indblik i, hvilke ting der er at se og opleve i den valgte by, med kommentarer og ratings fra andre brugere, der har besøgt disse steder.
Appen har også nogle mangler, som fx at vælge flere seværdigheder på listen, og give en rute mellem disse seværdigheder, så det er muligt at få en flerpunktsrute. \newline
\subsection{FindTheBestRoute.com}
Google Maps er begrænset til kun at kunne vise vejen fra et punkt til et andet. Det har FindTheBestRoute.com taget kampen op imod, og har derfor lavet en hjemmeside på FindTheBestRoute.com, hvor den hurtigste rute mellem maksimalt 10 forskellige adresser kan beregnes. FindTheBestRoute.com, udnytter Google Maps JavaScript API v3, altså en grænseflade til Google Maps, der tillader andre programmer at benytte Google Maps, til fx at få vist et kort, eller beregne en rute \citep{ftbr}.\newline
Selvom der på maps.google.dk ikke er mulighed for at indtaste forskellige destinationer, og få anvist den hurtigste rute imellem punkterne, så har Google Maps faktisk allerede funktionaliteten indbygget til at foretage denne beregning, baseret på ”The Travelling Salesman Problem”.\newline
For findthebestroute.com, er det derfor simpelt at sende en anmodning til Google, der indeholder informationer om de forskellige destinationer der skal forbindes med en rute. Google foretager så beregningerne, og sender den bedste rute tilbage til findthebestrute.com, hvor de så kan vise ruten til deres brugere \citep{googleapi}. \newline
\subsection{Opsummering}
TripAdvisor har mange gode funktioner, så som offline kort, "Best in Town" og information om de enkelte attraktioner. TripAdvisor findes som en app, så den er tilgængelig på ferien. TripAdvisior har dog ikke mulighed for at lave en flerpunktsrute. \newline
FindTheBestRoute.com har intet af det som Tripadvisor har, den har dog muligheden for at lave en flerpunktsrute, med maksimalt 10 adresser. Problemet med FindTheBestRoute.com er at den kun er tilgængelig på internettet og derved er den  mere besværlig på ferien.\newline
Disse synspunkter kan bruges til at udforme krav til projektets løsning. 