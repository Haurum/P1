\section{Interviewet}
På baggrund af interessant analysen kontaktede gruppen VisitAalborg, for at lave et interview vedrørende turisme I Aalborg. Dette interview blev forberedt som et semistruktureret interview, med en passiv spørgeteknik. Formålet med interviewet var, at fremskaffe viden og information fra en proffesionel kilde, der selv arbejder med emnet.
 
\subsection{Udformning}
Gruppen valgte at interviewet skulle være et semistruktureret enkeltinterview med passiv spørgeteknik. Herved kan respondenten snakke frit, så der er mulighed for længere uddybende svar, som ikke var tiltænkt af gruppen på forhånd. Disse teknikker blev brugt, da gruppen ikke havde meget viden om emnet på daværende tidspunkt, og derfor gav dette mulighed for mere information, end der var tiltænkt. For mere information om spørgeteknikkerne, se appendix B1.

Gruppen udformede en interview-guide, som kunne bruges under interviewet. Da respondenten var meget snaksagelig, blev denne guide ikke brugt til fulde. Ikke alle forberedte spørgsmål blev stillet, dette var dog ikke et problem, da der kom mange nye spørgsmål til, som interviewet udformede sig.  Spørgsmålene anvendt i interview-guiden, blev udformet efter gruppens initierende problemstilling og tilhørende underspørgsmål.
\subsection{Resultatbehandling}
Lars Bech nævner i interviewet, at den mest anvendte metode til bestemmelse af antal turister i Aalborg, er gennem målinger foretaget af Horesta, som gøres tilgængelig gennem Danmarks Statistik. Dette kan VisitAalborg bruge til at se fremgang i turismen i Aalborg. Hvis der ønskes resultater til antallet af turister, der overnatter i Aalborg, kan Danmarks Statistik. Dog gør Lars opmærksom på, at der også findes turister, som ikke overnatter, hvilket gør det svært, at finde det reelle antal turister, der besøger Aalborg.
VisitAalborgs medieudvikler Kim Mikael Jensen oplyser, at en udvidet løsning lignende TripAdvisor ville være interessant for både turisten og VisitAalborg, samtidigt nævner han, at de ikke kender til en lignende løsning endnu. Udfra dette kan der bekræftes, at der er interesse for en løsning. Dog gør Kim opmærksom på, at tidligere forsøg på løsninger, har været for komplicerede og har for mange funktioner. Han søger simplicitet i et program.

Der bliver i interviewet nævnt, at VisitAalborg er delvist kommunalt ejet. Dette viser, at staten har en interesse for turismen, hvilket stemmer overens med gruppens interessentanalyse.

Under interviewet diskuterer Lars, hvilke attraktioner der er interessante, ifølge tal, og kommer også ind på, hvad han selv mener er attraktioner i Aalborg. Som eksempler nævner han blandt andet IKEA, Aalborg Zoo og kunstmuseet, hvor han selv også mener, at gågaden, havnefronten og danske butikker (Georg Jensen, Inspiration osv). ”Hvis man køber et eller andet, som man er glad for, så kan man altid huske hvor man har købt det henne.” Dette citat, taget fra interviewet med Lars, passer overens med citatet fra interessentanalysen.
\subsection{Opsummering}
Igennem interviewet blev dele af problemstillingen bekræftet i form af, at et eventuelt program er ønsket. Der blev desuden foreslået krav til en eventuel løsning. Kim Mikael Jensen sagde, at programmet skal være simpelt, den skal altså ikke have for mange funktioner. Lars gjorde det også mere klart, hvad nogle af de populære attraktioner er, i Aalborg, fx IKEA, Aalborg Zoo og kunstmuseet. Interviewet er også blevet brugt til bestemmelse af begrebet ”attraktion”. 