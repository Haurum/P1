\chapter{Test}
For at sikre at programmet kører stabilt, og laver de korrekte udregninger, testes programmet for forskellige testcases. Der er forskellige metoder at teste dette på, hvoraf testen i denne rapport er blackbox testing. Denne testtype behandler en mængde testcases, som beskrevet i tabellen, hvor et bestemt input har et forventet output, og programmet testes derefter. Hvis det forventede output stemmer overens med det reelle output fra programmet, kan testen konkluderes succesfuld. 
En anden testtype kan være CU-test, som er et indlejret test-system, hvor testcases bliver oprettet som kode i programmet. Derefter vil testen blive kørt ved compile-time og resultater fra testcases bliver printet ud. Begrundelsen for brug af blackbox testing er, at der i dette program bliver printet gennem processen, og alle valg i programmet bliver foretaget af brugeren. Blackbox testing vil derfor give et godt billede, af om testen og input fra brugeren stemmer overens.

\section{Testcases}
Casene er opbygget således:\newline
\begin{tabular}{|l|l| p{5cm}|}
	\hline
	Case 1 & Korrekt valgt af attraktioner & Korrekt tilføjelse af interessante punkter \\ \hline
	Case 2 & Korrekt valgt af attraktioner & For mange tilføjelser af interessante punkter \\ \hline
	Case 3 & Korrekt valgt af attraktioner & Forkert input-type ved tilføjelser af interessante punkter\\ \hline
	Case 4 & Korrekt valgt af attraktioner & Ingen tilføjelse af interessante punkter (input ”0”) \\ \hline
	Case 5 & Valg af flere attraktioner end muligt & Forventer ikke prompt for andet input\\ \hline
	Case 6 & Valg af samme attraktion flere gange &	Forventer ikke prompt for andet input\\ \hline
	Case 7 & Forkert input-type til valg af attraktioner & Forventer ikke prompt for andet input\\ \hline
	Case 8 & Intet valg af attraktion (første input ”0”) & Forventer ikke prompt for andet input\\ \hline
\end{tabular} 
\newline

\textbf{Case 1:} \newline
I denne første testcase, vil inputtet til valg af attraktioner være 1, 5 og 9. Ved brug af disse tal, vil et forventet output være Aalborghus\_Slot for 1, Springeren\_-\_Maritimt\_Oplevelsescenter for 5 og Nordkraft for 9. Herefter ville valget af attraktioner afsluttes, ved input 0.
Output ved første del af testcasen blev følgende: ”Tilfoejet attraktion: Aalborghus\_slot”, ”Tilfoejet attraktion: Springeren\_-\_Maritimt\_Oplevelsescenter” og ” Tilfoejet attraktion: Nordkraft”. Herved gav den første del et korrekt output. Ved afsluttelse af valg af attraktion, tilføjes disse attraktioner til ruten, og næste trin er tilføjelse af interessante, nærliggende attraktioner. Eftersom de valgte attraktioner alle ligger tæt på havnen i Aalborg, vil andre interessante attraktioner være Utzon Centeret, Havnefronten og Friis, da disse alle ligger tæt på en tiltænkt rute fra Nordkraft til Springeren, hvor Aalborghus Slot også besøges.\newline
Her foreslår programmet følgende: 1: Utzon\_Centeret, 2: Friis\_Aalborg\_Citycenter og 3: Havnefronten. Dette er et korrekt output efter de attraktioner der blev valgt. Disse er alle tre inden for en afstand af 100 meter fra enten den nuværende rute eller de valgte attraktioner. Efterfølgende skal brugeren selv vælge, om han vil tilføje disse attraktioner til ruten. I dette tilfælde bliver inputtet 1 og 2, for tilføjelse af Utzon\_Centeret og Friis\_Aalborg\_Citycenter. Outputtet blev ”Tilfoejet attraktion: Utzon\_Centeret” og ”Tilfoejet attraktion: Friis\_Aalborg\_Citycenter”. Dette stemmer overens med det forventede output, og tilføjelsen afsluttes med input 0. Herefter vil ruten blive dannet, og alle attraktioner valgt vil blive printet ud som ”Din rute”. Heraf vil der vises Aalborghus Slot, Springeren, Nordkraft, Utzon Centeret og Friis. Disse vil sorteres efter hvornår på ruten de besøges, hvor startpunktet vil blive printet dobbelt, som både start-attraktion og slut-attraktion. Siden startattraktionen er Aalborghus\_Slot, skal ruten blive Aalborghus\_Slot, Utzon\_Centeret, Friis\_Aalborg\_Citycenter, Nordkraft, Springeren\_-\_Maritimt\_Oplevelsescenter og Aalborghus\_Slot. Dette er også tilfældet, da outputtet er magen til det forventede:\newline
Din rute:\newline
Aalborghus\_Slot\newline
Utzon\_Centeret\newline
Friis\_Aalborg\_Citycenter\newline
Nordkraft\newline
Springeren\_-\_Maritimt\_Oplevelsescenter\newline
Aalborghus\_Slot\newline
Herefter er der også et output der beskriver rutens længde, hvor et forventet resultat er udregnet gennem movable-type.co.uk, hvilket afrundet er 5.59km. Ifølge ouputtet er længden 5.61km, hvilket er omkring 20 meter fra det forventede resultat. Dette er et fint resultat, som viser, at programmet i dette tilfælde har en fejlberegning på 0.31\%. Denne fejlmargin er fin, da tallene i dette eksempel er afrundet.

\textbf{Case 2:} \newline
Der bruges samme input i denne case, derfor vil testen være den samme, indtil tilføjelsen af interessante attraktioner skal have input. Dette input testes med et input der er højere end antallet af forslag, hvilket i dette tilfælde vil være 4. Her kommer en fejlmelding fra programmet: ”Tallet svarer ikke til en attraktion”, og der promptes efter nyt validt input. Ved indtastning af samme tal flere gange, kommer den forventede fejlmelding ”Du har allerede indtastet denne attraktion. Proev igen”.

\textbf{Case 3:} \newline
Ligesom i case 2, er inputtet det samme indtil tilføjelsen af interessante attraktioner, hvor der i denne testcase testes for input af bogstaver, tegn og ord. I dette tilfælde vil ”a”, ”!” og ”test” alle tre printe ”Fejlindtastning – proev igen”, og derefter promte efter nyt validt input. I en tidligere version printede programmet denne sætning for hvert tegn og bogstav inputtet bestod af. Så i ”test” blev der printet fire ”Fejlindstastning – proev igen”. 

\textbf{Case 4:} \newline
Igen her blev testen udført med det samme input som i case 2 bortset fra, at inputtet til tilføjelsen af interessante attraktioner vil være ”0”, for afsluttelse af ruten uden tilføjelse af attraktioner. Her kører programmet videre, og giver den endelige rute: \newline
Din rute:\newline
Aalborghus\_Slot\newline
Nordkraft\newline
Springeren\_-\_Maritimt\_Oplevelsescenter\newline
Aalborghus\_Slot\newline
Rutens længde ville forventet afrundet være 5.54 km, og outputtet fra programmet siger 5.56 km, som igen er omkring 20 meter længere.

\textbf{Case 5:} \newline
Ved første prompt viser den antallet af attraktioner, og hvis alle attraktioner vælges, bliver der ikke promptet for en attraktion ud over det maksimale antal af attraktioner. Dette vil sørge for, at en bruger ikke kan vælge flere attraktioner end databasen er tilskrevet. Programmet vil derfor fortsætte videre til valg af alle attraktioner.

\textbf{Case 6:} \newline
I denne testcase vil det første prompt testes for, hvorvidt det er muligt at indtaste den samme attraktion flere gange. Forventningen er, at en fejlmelding forekommer ved mere end én indtastning for samme attraktion. Ved indtastning af 1 to gange i træk, kom fejlmeldingen ”Du har allerede indtastet denne attraktion. Proev igen”. Derefter testes for, hvorvidt der kan skrives to forskellige tal, hvoraf det første bliver skrevet to gange, med det andet tal i mellem, dvs. 1, 2 og 1. Her forventes samme fejlmelding ved anden indtastning af 1 to gange i træk. Programmet printede den samme fejlmelding.

\textbf{Case 7:}\newline
I testcase 7 vil det første prompt igen blive testet, denne gang for input af tegn, bogstaver og ord. Igen testes med ”a”, ”!” og ”test”. Her forventes samme resultat som i testcase 3, hvor programmet i dette tilfælde vil printe fejlmeldingen ”Fejlindstastning – proev igen”. Her var resultatet som forventet, i alle tre tilfælde blev der printet ”Fejlindstastning – proev igen”. Ved  indtastning af et tal højere end antallet af attraktioner, forventes fejlmeldingen ”Tallet svarer ikke til en attraktion”. Dette er også tilfældet, da programmet giver den rigtige fejlmelding.

\textbf{Case 8:}\newline
I denne case testes programmet for intet valg af attraktioner, ved at første input er 0. Forventningen i denne test er, at programmet blot afsluttes. Dette er også tilfældet, dog havde en rettelse været nødvendig, da det originale program blot udskrev en rute uden attraktioner, med en afstand på 100.000 km.


\textbf{Opsamling}\newline
Efter disse testcases kan det påvises, at programmet kører som planlagt, efter et par rettelser. Der er testet for, hvorvidt programmet giver en rute ved korrekt input, muligheder for forkert input, hvorvidt programmet giver en korrekt rutelængde, samt om attraktionerne tilføjet som interessante attraktioner er korrekte. Brugeren skulle ikke have mulighed for at give forkert input, rutelængden er med minimal fejlagtighed, korrekte interessante attraktioner bliver tilføjet, og ved et korrekt input vil en rute altid blive oprettet.
