\chapter{Problembeskrivelse}

I dette afsnit vil den endelige problemformulering blive beskrevet.

\section{Problemafgrænsning}
Ud fra denne analyse, kan der konkluderes, at VisitAalborg og turisterne er ressourcepersoner, hvoraf turisterne har større indflydelse på programmet, da de er de endelige brugere. VisitAalborg kan bruges som guider til, hvad der kan være af indhold i programmet, men der skal stadig tages højde for, at det er i deres interesse, at få deres arbejdspartnere med ind i programmet, selvom det ikke i alle tilfælde er til turistens interesse.
Ved spørgeskemaet blev der uddraget, at turister allerede har løsninger fra punkt til punkt, hvor der i eksisterende løsninger blev påpeget, at der også findes løsninger for flerpunktsruter. TripAdvisor viste, at der også er programmer, som tager højde for brugerens interesse, men alt taget i betragtning, er der ikke en løsning der kombinerer alle disse funktioner, som der belyses i gruppens spørgeskema og interview, til at være i brugerens bedste interesse: En simpel løsning, der inddrager brugerens interesse, og foreslår yderligere punkter til en mere interessant rute. Dette kunne endda optimeres ved en offline-funktion, som TripAdvisor også gør brug af ved et offline kort.

\section{Problemformulering}
Hvordan udvikles der en softwareløsning, der hjælper turisten med at finde rundt i en storby, på en interessant rute mellem turistens egne valgte attraktioner?
\begin{itemize}
	\item Hvordan kan programmet finde en rute?
	\item Hvordan kan programmet gøre ruten interessant?
	\item Hvordan kan programmet hjælpe brugeren med at finde rundt?
\end{itemize}
