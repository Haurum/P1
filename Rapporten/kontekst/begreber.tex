\subsection{Begrebsdannelse}
Ved en ’interessant rute’ vil der i denne rapport refereres til en rute, der indkluderer unikke kulturelle og nationale oplevelser, som dog er individuel for turisten. I appendix (interview) sagde Lars fra VisitAalborg, at interessante attraktioner i aalborg blandt andet er Aalborg Zoo, Utzon centeret, kunstmuseet, gågaden, havnen og City Syd. En rute der indkluderer danske traditioner og spise. \newline

En ’attraktion’ vil i denne rapport forståes som både traditionelle, og utraditionelle attraktioner. Traditionelle attraktioner i Aalborg er Aalborg Zoo, kunstmuseet og lignende. De utraditionelle indkluderer steder som City Syd, Ikea og Jensens Bøfhus. \newline 

’Turist’ vil i denne rapport være til dels defineret ud fra UNWTOs (UN World Tourism Organistation) beskrivelse, som beskriver to slags turister. En-dagsturister, som maksimalt overnatter én nat på stedet, og en generel turist, som overnatter flere end en nat. Udover disse to typer ser vi også en undergruppe ’erhvervsturist’. Dette er turister, der kommer til stedet med et arbejdsformål. 
En turist er en som har et bestemt formål med rejsen, til forskel for besøgende og rejsende. Formålet kan fx være, at opleve den danske kultur.  \newline

En ’storby’ i Danmark, er i denne rapport en by, omgivet af områder med lavere bebyggelse.
