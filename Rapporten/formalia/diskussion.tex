\chapter{Diskussion}
I dette kapitel vil gruppen opsamle på de beslutninger, som er blevet truffet gennem projektet. Samtidig vil gruppen se på fejlkilder der er opstået og hvilken indflydelse disse fejlkilder har haft på projektet. 

\section{Spørgeskema}
Tidligt i projektet sendte gruppen et spørgeskema ud på deres facebookprofiler, som i alt gav 60 besvarelser. Da gruppen delte det på deres facebookprofiler, ville det betyde at gennemsnitsalderen på respondenterne, ville være relativ lav. Gruppen havde en række spørgsmål som vi gerne ville have svar på, og derfor blev spørgeskemaet opbygget af mange underspørgsmål, som var blevet lavet i forhold til den initierende problemstilling. Spørgeskemaet konstaterede nogle problemer, som fx at turisterne generelt havde svært ved at finde rundt, og foretrak en interessant rute, frem for en hurtig rute. 

Turister var sat som ressourceperson i interessentanalysen, da det var dem som programmet var rettet imod, dog mener gruppen at spørgeskemaet blev sendt for hurtigt ud, og uden eftertanke. Med det menes der, at mange af spørgsmålene ikke var ordentlig formuleret. Herunder kan der ses et eksempel på et dårligt formuleret spørgsmål, som gruppen sendte ud: \newline
\begin{itemize}
	\item “Har du nogensinde haft problemer med at finde vej på din storbyferie?” 
\end{itemize}
Dette eksempel, mener vi, er dårligt formuleret, da det er alt for åbenlyst hvad brugeren vil svare, for hvem har ikke haft problemer på en ferie? Her burde vi have været mere specifikke i spørgsmålet, som fx “I hvilke situationer har du haft problemer med at finde vej i en storby?”, med mulighed for at skrive eget svar i stedet for valgmuligheder. \newline
Selve projektet er rettet mod turister i Aalborg, hvor 41\% af turisterne er udenlandske, så da gruppen havde valgt at skrive spørgeskemaet på dansk, er det ikke hele målgruppen der rammes. Samtidig med dette, var spørgeskemaet som tidligere nævnt, delt på gruppemedlemmernesderes facebookprofiler, hvilket også begrænser respondentgruppen. De fleste respondenter vil derfor være dansktalende, hvilket i denne situation ikke var et problem, idet spørgeskemaet var skrevet på dansk. For at få flere respondenter, og specielt udenlandske respondenter, burde gruppen have delt spørgeskemaet flere steder end bare på Facebook, samt lavet spørgeskemaet på både engelsk og dansk. 

Gruppen skulle have brugt mere energi/tid på spørgeskemaet, og med eftertanke burde vi nok have udsendt et nyt spørgeskema. Selve spørgsmålene skulle være bedre gennemtænkte og ikke så ledende som de var blevet skrevet. Ud fra spørgeskemaet blev der dog konstateret et problem, som projektet kunne tage udgangspunkt i. Respondentgruppen er dog relativ lille, og det ville have været bedre at få en både større, men også bredere respondentgruppe. Med ordet bredere mener gruppen at der ønskes både respondenter fra ind- og udland i forskellige aldre.

\section{Interview}
I dette projekts interessentanalyse blev turistbureauer, i dette tilfælde VisitAalborg, sat som ressourceperson, da de kunne bidrage med informationer om turisme i Aalborg. Gruppen besluttede derfor at skaffe et interview med VisitAalborg, hvor gruppen fik fat i Lars Bech og Kim Mikael Jensen, som gerne ville stille op til et interview. \newline
Gruppen havde udarbejdet en interviewguide, som var lavet i punktform, der beskrev hvilke tanker vi havde om emnet, og hvilke spørgsmål vi gerne ville stille Lars og Kim. Interviewet var planlagt som et ustruktureret interview, så hvis Lars/Kim ikke var meget for at snakke, eller ikke kom frem med det, vi efterspurgte, ville vi kunne spørge mere ind til emnet. Dette var dog ikke tilfældet med Lars, da han snakkede rigtig meget. Han førte ofte interviewet i en anden retning, end hvad vi havde planlagt. Eftersom vi i gruppen ikke rigtig havde lavet et interview før, var interviewerne ikke så gode til at stoppe ham, når han tog spørgsmålet i en anden retning. Dette gjorde at interviewet udviklede sig til, at Lars nærmest tog styringen af interviewet, men det gav mulighed for at få information, vi ikke selv havde overvejet, forud for interviewet.

Ud fra interviewet med Lars og Kim, blev gruppen klogere på hvilke turister der oftest besøger Aalborg, og hvilke attraktioner der er populære i Aalborg. Lars og Kim virkede interesserede i projektet, de havde dog tidligere arbejdet med en elektronisk løsning, men de var blevet nødsaget til at gå tilbage til kort og brochure. 

Interviewet, var ligesom spørgeskemaet, udført uden den store eftertanke. Dette gjorde, at interviewguiden ikke blev så god, som gruppen havde håbet, og vi skulle i stedet have ventet til lidt senere i forløbet med at udføre interviewet, så gruppen havde mere konkret viden om emnet, og om hvilke spørgsmål, gruppen ville spørge professionelle på emnet om. 

\section{Programmet}
Programmet er lavet som beskrevet i afsnittet ”Implementering”. Gruppen havde fra begyndelsen af løsningsdelen, valgt at lave løsningen i fugleflugtslinje. Dette er dog ikke det mest optimale, da der kan forekomme problemer i forhold til vejnettet i Aalborg, samt at der kan være bygninger eller andre forhold, der ikke medregnes, når der bliver målt i fugleflugtslinje. Dette har dog simplificeret programmet, i forhold til hvordan det laves, men det har også givet en usikkerhed, når den korteste rute skal bestemmes. Gruppen kan ikke garantere, at den givne rute i realiteten er den korteste, når der også skal tages højde for hvilke veje der rent faktisk kan benyttes. \newline
Hvis gruppen skulle have implementeret en løsning, der tager højde for vejnettet, havde gruppen tænkt på to forskellige løsninger. Den første ville være at sætte hele Aalborg op i et grid, hvor vejnettet ville blive markeret med 1 og resten ville være markeret med 0. På den måde ville vejen kunne findes med forskellige søgealgoritmer, som fx A*. En anden løsning ville være at have en tabel, der indikerer hvilke veje der er forbundet og distancen der mellem. Disse løsninger ville have gjort ruten mere præcis, da den reelle korteste rute ville kunne findes. 

For at beregne den korteste rute mellem attraktionerne, valgte vi i gruppen at benytte Nearest neighbour algoritmen, på grund af dens hurtige eksekveringstid, der gør det muligt for brugeren af programmet, at indtaste et højt antal attraktioner, uden at det går mærkbart ud over oplevelsen med programmet. Algoritmen blev også valgt, på grund af den forholdsvis simple implementering af den, og at den generelt passede godt til vores behov. Problemet ved at bruge denne algoritme, er at den ikke nødvendigvis finder den korteste rute. Algoritmen er upræcis, og en fejlmargen bliver nødt til at accepteres, hvis ikke man vælger at gribe ind overfor algoritmen, hvis den er ved at gøre noget der tydeligt giver en længere rute end nødvendigt. Den korteste rute burde fx aldrig krydse sig selv, og det var muligvis en af de ting vi kunne have tjekket for, når algoritmen benyttes, for at sikre os at den i det mindste ikke gør det, og på den måde får en kortere rute, end hvis vi bare havde ladet den gennemføre sine beregninger. 

En anden løsning, ville være at prøve alle ruter der overhovedet er for de valgte attraktioner, men eksekveringstiden stiger faktorielt med antallet af attraktioner, så der skal ikke vælges mere end et par stykker, før brugeren af programmet begynder at kunne mærke at det tager lang tid at lave beregningerne. Et scenarie vi helst gerne ville undgå.
Double Minimum Spanning Tree og Dijkstra’s var også algoritmer gruppen havde undersøgt, og prøvet at implementere, men da fokuset ikke var på at lave den korteste rute, men derimod give en interessant rute, blev simplicitet og hurtig eksekveringstid prioriteret. Derfor endte vi med at bruge Nearest neighbour algoritmen.
