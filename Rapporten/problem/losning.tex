\chapter{Problemløsning}
I dette kapitel vil gruppen beskrive hvilke krav gruppen stiller til projektet, hvilke teorier og algoritmer der er vigtige i forhold til projektet og hvordan gruppen har valgt at lave programmet, altså implementeringsprocessen.

\section{Kravspecifikationer}
I dette afsnit vil gruppen vurdere, hvilke krav der skal indegå i en løsning. Heri vil der både blive opstillet krav til en optimal løsning og til en afgrænset løsning, som gruppen mener at være realistisk, at kunne lave.

\subsection{Optimale løsningsforslag}
For at kunne udvilke en softwareløsning, der besvarer gruppens problemformulering, er det vigtigt at definere nogle krav til programmet. Til en optimal løsning, har gruppen vurderet, at der skal være følgende krav:
\begin{itemize}
	\item Programmet skal udvikles som en applikation til moderne smartphones, inklusiv iOS, Android og Windows. Phone. 
	\item Programmet skal vise et kort med ruten, og kortet skal vise attraktioner der er tæt på. 
	\item Programmet skal kunne beregne den korteste rute mellem en række punkter.
	\item Programmet skal vise rutevejledningen på samme måde som normale GPS-enheder. 
	\item Programmet skal give mulighed for at bedømme attraktioner, og derved tildele dem en rating, så de mest populære attraktioner kan findes. 
	\item Programmet skal give forslag til en anden rute, der inkludere attraktioner der ligger tæt på ruten.
	\item Programmet skal kunne  downloade en offline version af den rute der er valgt, inklusiv kortet for det omkringliggende område.
\end{itemize}
Til disse krav har gruppen konstrueret nogle skitser af den optimale løsning, for at give et billede af hvordan det eventuelt kunne se ud. \newpage

\begin{wrapfigure}{r}{0.3\textwidth}
  \vspace{-20pt}
  \begin{center}
    \includegraphics[scale=0.35]{start1} \newline
    \textit{Figur 3.1: Brugergrænseflade}\newline
  \end{center}
  \vspace{-20pt}
  \vspace{-20pt}
\end{wrapfigure}


Gruppen ønsker, at programmet skulle fungere på den måde, at der findes to valg muligheder, forholdsvis rating og afstand, hvoraf rating viser en række attraktioner med en værdi, baseret på hvad brugerne har valgt at rate den. Funktionen afstand, vil vise hvor stor en afstand der er fra det punkt hvor brugeren står, til en attraktion. De attraktioner, som brugeren ønsker at se, skal brugeren blot tjekke af, ved at klikke på attraktionerne, og de vil derefter blive tilføjet til den nuværende rute. Figur 3.1 viser en skitse af brugergrænsefladen. \newline
\newline
\newline
\newline

\begin{wrapfigure}{l}{0.3\textwidth}
  \vspace{-50pt}
  \begin{center}
    \includegraphics[scale=0.35]{start2} \newline
    \textit{Figur 3.2: \newline Udvidet brugergrænseflade}\newline
  \end{center}
  \vspace{20pt}
\end{wrapfigure}

Herudover ønsker gruppen, at der er en form for menu, som indeholder informationer omkring de forskellige attraktioner. Udover dette, skal der også være mulighed for at vælge et start- eller slutpunkt. Disse punkter skal give brugeren mulighed for at vælge, hvor brugeren ønsker at starte/slutte sin rute. Figur 3.2 viser en skitse af en udvidet brugergrænseflade.\newline
\newline
\newline
\newline
\newline

\begin{wrapfigure}{r}{0.3\textwidth}
  \vspace{-30pt}
  \begin{center}
    \includegraphics[scale=0.3]{rute1} \newline
    \textit{Figur 3.3: Rute - Ruten er\newline hentet fra Maps.Google.com}\newline
  \end{center}
  \vspace{0pt}
  \vspace{-100pt}
\end{wrapfigure}


Når der er valgt nogle ønskede destinationer/attraktioner, skal programmet fremvise en rute. Ruten skal vise hvor lang hele ruten er, og hvor lang tid det tager at gå ruten. Der skal desuden være to funktioner, når ruten bliver vist. Der skal være mulighed for at downloade kortet på mobilen, og derved gør det muligt at anvende programmet, uden brug af internet. Den anden funktion skal starte rutevejledningen, som fungere som en ganske almindelig GPS. En skitse af dette kan ses på figur 3.3.
\newpage

\begin{wrapfigure}{r}{0.3\textwidth}
  \vspace{-20pt}
  \begin{center}
    \includegraphics[scale=0.3]{rute2} \newline
    \textit{Figur 3.4: Udvidet rute - Ruten er hentet fra\newline Maps.Google.com}\newline
  \end{center}
  \vspace{-20pt}
  \vspace{-10pt}
\end{wrapfigure}

Der skal desuden også her være mulighed for at få information om attraktionerne, der er i nærheden af den valgte rute. Der skal være en "tilføj til nuværende rute"-funktion, som tilføjer de valgte attraktioner, som er i nærheden til den rute, der allerede er lavet. Figur 3.4 viser hvordan det eventuelt kunne laves. \newline
\newline
\newline
\newline
\newline
\newline
\newline
\newline
\newline
 

\subsection{Gruppens løsningsforslag}
Gruppen har gennem spørgskema og interview, fået stillet en række krav til løsningen, af turister og VisitAalborg. 
Gennem spørgskemaet, blev det konkluderet, at det vigtigste for turister, er at de kan opleve byen på en interessant rute. 
Derudover har turistbureauet givet udtryk for, at løsningen gerne skal være så enkelt som muligt, altså meget få funktioner, så brugeren ikke bliver forvirret, da de mener, at det er i turistens bedste interesse. \newline
Der er blevet stillet krav fra universitets side, om at programmet skal være et lille specifikt program i C, af høj kvalitet. Dette stemmer godt overens, med de krav der er blevet stillet fra turistbureauets side.   \newline
Ud fra dette, har gruppen opsat nogle krav for gruppens løsningsforslag, og de er som følgende:
\begin{itemize}
	\item Programmet skal kunne beregne den korteste rute mellem en række punkter.
	\item Programmet skal være i stand til at give forslag til en anden rute, der inkludere attraktioner der ligger tæt på ruten.
 	\item Programmet skal som output, give en liste over rutens destiationer, sorteret efter tiden til attraktionerne.
\end{itemize}

Da dette er et P1 projekt, og gruppen er begrænset af både tid og erfaring, har gruppen valgt at begrænse softwareløsningen, på følgende punkter: 
\begin{itemize}
	\item Rutevejledningen bliver i fugleflugtslinje.
	\item Brugeren kan kun vælge destinationer ud fra en række forudbestemte punkter.
	\item Tekstbaseret brugergrænseflade.
\end{itemize}

På baggrund af kravene og afgrænsningen, har gruppen tænkt sig at lave et program, som har nogle forudbestemte destinationer, der dækker over destinationerne i Aalborg, hvorefter brugeren vælger de destinationer han/hun ønsker at besøge. Programmet vil ud fra disse punkter, beregne den korteste rute, og undersøge om der er andre attraktioner, som ligger tæt på ruten, og spørge brugeren, om det kunne være interessant at besøge disse steder. Hvis ja, vil disse punkter også blive inkluderet. På den måde for brugeren selv lov til at skabe sig den mest interresante rute. Resultatet bliver en liste over destinationerne, der står i rækkefølge, så turisten ved hvilken rækkefølge de skal besøge dem i, for at få den mest optimale rute.
