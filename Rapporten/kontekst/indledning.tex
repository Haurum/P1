\chapter{Indledning}

Hvert år besøger flere millioner turister Danmark, hvilket er godt for den danske økonomi. Når turisterne bruger penge på en dansk varer, service eller oplevelse, bliver det sådan set eksporteret til udlandet – og derfor bliver dette betragtet som en eksportvare. I alt står denne eksport type for 3,6\% af den danske eksport. Turisterne har et forbrug på 87,2 mia. kr., hvoraf de udenlandske turister bruger 35,7 mia. kr. altså godt 41\%, mens de danske turister står for de resterende 59\%. Udover at turismen hjælper det danske samfund økonomisk, skaber turismen ifølge VisitDenmark knap 122.500 fuldtidsjobs. \citep{faktaogtalVD}   \newline
Det ses gerne at turisterne kommer tilbage til Danmark igen. Dette sker naturligvis ved at turisterne nyder deres ophold og får den bedst mulige ferie. Som turist i en storby kan det forekomme svært at finde rundt, og kan let fare vildt. Hvis en turist i Paris gerne vil se Eiffeltårnet, skal turisten blot kigge efter det store monument, og gå i den retning hvor attraktionen nu er. Dog kan det ske at turisten undervejs mister Eiffeltårnet af syne, og pludseligt ved turisten ikke i hvilken retning personen nu skal gå. Turisten kan vælge at bruge sin smartphone, hvis turisten da er i besiddelse af en, og kan eksempelvis gå på internetsiden GoogleMaps. Dog kan brugen af GoogleMaps være en dyr fornøjelse, da mobildata i udlandet koster en del penge. Derfor vil en offline løsning være billigere løsning for turisterne. Her kan turisten så finde en rutevejledning fra punkt A til B, dog vil der kunne opleves problematikker, hvis en flerpunkts rute ønskes. Dette behøver ikke kun at ske i storbyer så som Paris, dette kunne også ske i nogle af de større danske byer. \newline
Turister vil generelt opdage, at planlægning på forhånd er en god ting, hvis turisterne vil nå så mange attraktioner som muligt på en ferie, da tiden kan være begrænset. Dog kan der samtidigt spørges, hvad der gør en rute god: Er det hvor hurtigt turisten kommer fra den ene valgte attraktion til den anden? Kan der findes en mere interessant rute, eventuelt med attraktioner der ikke er oplyst i rejsebureauets brochurer, måske en smutvej forbi havnen eller muligheden for en flerpunkts rute mellem attraktionerne? Der kan være mange parametre der spiller ind, når den foretrukne rute skal vælges. \newline
Hvis en rejse skulle gøres mere interessant for turister, kan der så udvikles et program, der hjælper turisten med at finde den foretrukne rute mellem attraktionerne? Hvad vil være den foretrukne rute: Den hurtigste, eller den mest interessante? Og i hvilket omfang vil dette program kunne hjælpe turisten? \newline
At lave et program der gælder for hele verdens storbyer, ville kræve meget mere programmerings erfaring og tid. Derfor har gruppen i dette projekt valgt at afgrænse sig til at arbejde med ruteplanlægning for turister i Aalborg. Der kan også være forskellige opfattelser for hvad hver person ser som en attraktion. Gruppen har derfor valgt at give nogle eksempler på hvad gruppen mener er attraktioner i Aalborg. Det kunne bl.a. være kulinariske oplevelser på diverse restauranter. Kulturelle attraktioner som Tyren, museer, havnen, koncerter og andre arrangementer. Det kunne være oplevelser som en tur i Zoo eller Aalborgtårnet. Steder som City Syd, gågaden og Jomfru Ane Gade mener gruppen også kan gå ind under attraktioner.

\section{Metoder}
Ud fra interessentanalysen, som kan findes i problemanalysen, har gruppen valgt at lave et spørgsekema og et interview. Spørgeskemaet blev delt af gruppen på Facebook, men den rettede sig dog kun mod danskere, da den var skrevet på dansk. Derudover er spørgeskemaet ikke begrænset til byen Aalborg, da størstedelen af den målgruppe, som besvarede spørgeskemaet, bor tæt på eller i Aalborg. \newline
Ved hjælp af spørgeskemaet besluttede gruppen at opstille et interview med VisitAalborg. Gruppen kontaktede derfor VisitAalborg via mail og blev henvist til Lars Bech. Interviewet blev derefter udført af to gruppemedlemmer. Hele interviewet er blevet transskriberet, som er at finde i Appendix B.2. 

\subsection{Den kvalitative metode}
Den kvalitative metode går ud på at få noget data med god kvalitet, som kommer direkte fra en kilde som har en erfaring eller viden omkring emnet. Dette skal hjælpe med at finde problemer til projektet. Dette sker ved en samtale hvor interviewer og respondent mødes ansigt til ansigt, på den måde kan man få nogle bedre uddybende svar i forhold til ja/nej-svar \citep{kvalitativ}. 
Denne metode er blevet anvendt i form af et interview. Gruppen har valgt at lave et interview, da gruppen allerede havde en delvis forståelse af emnet. Dog ville gruppen have en mere professionel tilgang til dette,  og om en eventuel løsning ville være brugbar.

\subsection{Den kvantitative metode}
Den kvantitative metode går ud på at få en masse data, fra en repræsentativ gruppe, for at få holdninger og lignende om det valgte emne. Disse holdninger kan så være med til at generalisere det output der fremkomer for målgruppen \citep{kvantitativ}. \newline
I form af den kvantitative metode har gruppen opstillet et spørgeskema, som gruppen valgte at dele på hver deres Facebook profiler. Grunden til at gruppen opstillede dette spørgeskema, skyldes at gruppen gerne ville se om der var interesse i dette projekt, og om det var et problem eller ej.
