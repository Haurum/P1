\chapter{Spørgeskema}

\section{Teori}
\subsection{Metoder}
Ud fra interessentanalysen, som kan findes i problemanalysen, har gruppen valgt at lave et spørgeskema og et interview. Spørgeskemaet blev delt af gruppen på Facebook, men den rettede sig dog kun mod danskere, da den var skrevet på dansk. Derudover er spørgeskemaet ikke begrænset til byen Aalborg, da størstedelen   af den målgruppe, som besvarede spørgeskemaet, bor tæt på eller i Aalborg.
Ved hjælp af spørgeskemaet besluttede gruppen at opstille et interview med VisitAalborg. Gruppen kontaktede derfor VisitAalborg via mail og blev henvist til Lars Bech. Interviewet blev derefter udført af to gruppemedlemmer. Hele interviewet er blevet transskriberet, som er at finde i Appendix B.2.

Den kvalitative metode \newline
Den kvalitative metode går ud på at få noget data med god kvalitet, som kommer direkte fra en kilde som har en erfaring eller viden omkring emnet. Dette skal hjælpe med at finde problemer til projektet. Dette sker ved en samtale hvor interviewer og respondent mødes ansigt til ansigt, på den måde kan man få nogle bedre uddybende svar i forhold til ja/nej-svar[1]. 
Denne metode er blevet anvendt i form af et interview. Gruppen har valgt at lave et interview, da gruppen allerede havde en delvis forståelse af emnet. Dog ville gruppen have en mere professionel tilgang til dette,  og om en eventuel løsning ville være brugbar. 

Den kvantitative metode \newline
Den kvantitative metode går ud på at få en masse data, fra en repræsentativ gruppe, for at få holdninger og lignende om det valgte emne. Disse holdninger kan så være med til at generalisere det output der fremkommer for målgruppen[2].
I form af den kvantitative metode har gruppen opstillet et spørgeskema, som gruppen valgte at dele på hver deres Facebook profiler. Grunden til at gruppen opstillede dette spørgeskema, skyldes at gruppen gerne ville se om der var interesse i dette projekt, og om det var et problem eller ej. \newline
[1] - http://www.analysedanmark.dk/kvalitative-metoder/ \newline
[2] - http://www.analysedanmark.dk/kvantitative-metoder/ 


\section{Resultater}
I dette afsnit vil overvejelserne i forbindelse med udformningen af spørgeskemaet og dets resultater blive beskrevet

\subsection{Udformning}
Først udformede vi en række problemstillinger vi ville have svar på i vores undersøgelse:
Hvilke slags attraktioner tager turister til Aalborg for at se?
Hvordan planlægger turister deres ferie til Aalborg?
Hvordan finder turister rundt?
Har turister problemer med at finde rundt?
Fortrækker turister den hurtigste eller den mest interessante rute?
Er den løsning vi har i tankerne noget respondenterne ville bruge?

Derefter kiggede vi på hvem vi skulle spørge. Da vi havde afgrænset os til Aalborg kom vi i problemer da der stort set ingen turister er i Aalborg om vinteren. Derefter tænkte vi at vi kunne spørge vores venner på Facebook, men ved nærmere eftertanke var der ikke mange af dem som har været decideret turister i Aalborg. Derfor kom vi frem til at vi ville lave spørgeskemaundersøgelsen for vores Facebook venner, men hvor der ikke blev afgrænset til Aalborg men til storbyer generelt. Dette ville give os nogle resultater for storbyturisme generelt som vi så kunne overføre til Aalborg, da det også er en storby. Dog giver dette nogle begrænsninger: Vi spreder spørgeskemaet på Facebook og vil derfor kun ramme de personer som vi er venner med på sitet. Det vil sige at vi hovedsageligt vil ramme unge mennesker mellem 19 og 23 år. Derudover har vi valgt at lave spørgeskemaet på dansk og vil derfor ikke ramme folk som ikke forstår dansk. Det er simpelthen gjort fordi spørgsmålene skulle være så nemme at forstå som muligt, og da vi ikke har ret menge venner på Facebook som ikke kan dansk, mister vi ikke ret mange respondenter ved at gøre dette.

Selve spørgeskemaet med svar er i appendiks A.2

\subsection{Resultatbehandling}
Først og fremmest er det vigtigt at være opmærksom på at 60 er en relativ lille respondentgruppe, men det giver stadig nogle klare tendenser, som vi godt mener vi kan tillade os at konkludere på.

Vi kan se  at, det at se byens seværdigheder, opleve kulturen og maden var de 3 ting de fleste valgte når vi spurgte hvad der var vigtigt for dem på storbyferie, de fik henholdsvis 90\%,  68.3\% og 63.3\%. Folk tager altså først og fremmest til storbyen for at se dens seværdigheder og dernæst opleve dens unikke kultur/stemning og/eller madoplevelser. Halvdelen vil gerne shoppe på deres storbyferie, så de forskellige shoppingmuligheder ligger i midten af hvad vores respondentgruppe gerne vil på deres storbyferie. Som de 3 ting på vores liste folk valgte mindst har vi Fest/Bytur, Teater/Musik og Museumsbesøg, med henholdsvis 26.6\%, 18.3\% og 11.6\%. De fleste i vores respondentgruppe tager altså ikke til storbyen for at opleve museer eller teater/musikforstilinger. Derudover blev vi gjort opmærksom på, at der er flere som tager til storbyen for at se Sportsbegivenheder. Dette havde vi dog ikke som mulighed på listen og vi kan derfor ikke komme med en ordentlig konklusion på hvor mange fra vores respondentgruppe som tager til storbyen for sportsbegivenheder.
Vi kan herudfra se at vi skal have fokus på disse 4 attraktioner i rækkefølge med den vigtigste først:

\begin{itemize}
	\item Byens serværdigheder
	\item Opleve kulturen
	\item Maden
	\item Shopping
\end{itemize}

Vi kan se at der ikke er nogle som planlægger deres ferie til punkt og prikke, men at stort set alle planlægger, enten nogle ting inden ferien og/eller dag for dag på ferien. Vi kan derfor konkludere at vores produkt, som vil være et planlægningsværktøj, passer ind i respondentgruppens allerede etablerede rutiner.
Vi kan også se at respondengruppen allerede bruger massere af værktøjer til at planlægge deres ferie (Tripadvisor, google(maps), ”turen går til” bøger, osv.), hvilket giver os et billede af de konkurrenter vi har.

Vi kan se at at 68.3\% af respondentgruppen har haft problemer med at finde rundt på deres storbyferie. Vi har altså en gruppe hvor 2 ud af 3 kunne have undgået deres problemer med bedre ruteplanlægning/vejledning.
Når respondentgruppen blev spurgt hvilke redskaber de brugte til at finde vej på deres storbyferie, fik vi svaret Diverse kort/brochure som klar nr. 1. De fleste bruger altså stadig gammeldags kort til at finde rundt i storbyen. Derudover brugte omkring halvdelen GPS/Elektronisk kort eller De lokale som redskab til at finde rundt. Et flertal bruger altså gammeldags kort i forhold til de nye elektroniske midler som er kommet til. 

Vi ser tydeligt at vores respondenter fortrækker den interessante rute over den hurtigste rute, med henholdsvis 80\% for den interessante og 20\% for den hurtigste. Folk er altså mere interesseret i at få flere oplevelser end at komme hurtigt frem til næste punkt på dagsordenen.

Sidste spørgsmål i spørgeskemaet lød således:
Et program/applikation, som hjælper mig med at finde den hurtigste og/eller mest interessante vej igennem byen, via mene valgte ”must see” destinationer, ville være noget jeg kunne bruge?
Hvilket over 90 procent svarede enig eller meget enig til. Vi kan altså konkludere at respondenterne i vores undersøgelse ville have gavn af at bruge vores løsningside. 
