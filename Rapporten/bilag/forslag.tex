\chapter{Nyt p1 forslag}
{\Huge\textbf{Grundlæggende programmering:\newline Søgemaskiner.}}\newline

\textbf{Beskrivelse:}\newline
På hvilke IT-C hjemmesider finder man information om f.eks. "fodbold", eller oplysing om "eksamensdatoer". Til at svare på et sådan spørgsmål benyttes en søgemaskine. En sådan søgemaskine består dels af teknologi der muliggøre at flere bruger systemet samtidigt (mange bruger f.eks. yahoo samtidigt), og teknologi der muliggøre hurtige søgninger. Disse teknologier kan man lære om i dybden på hhv. webprogrammerings- og algoritmekurser på IT-C. Formålet med dette projekt er at konstruere en simpel søgemaskine. I projektet begrænser vi os til forudsætningerne fra grundlæggende programmering. Hensigten er at projektet skal munde ud i et simpelt program som kan:

\begin{enumerate}
\item Indlæse en fil med en masse ord, som "abemor", "katte" og "husvogn".\newline
\item En simpel brugergrænseflade hvorfra der kan spørges om et givet ord var i filen.
\end{enumerate}

For at kunne besvare et spørgsmål hurtigere end at undersøge samtlige ord fra filen, starter projekter med en enkelt forklaring i hvorledes man kan kostruere en simpel hurtig søgerutine.
Forslag til varianter af opgaven er velkommen. Formålet med projektet er at de studerende skal opnå erfaring med at konstruere et større program ved at sammensætte og anvende de ting der er blevet undervist i på kurset grundlæggende programmering. Denne erfaring er en stor støtte for en bred vifte af efterfølgende kurser på IT-C: F.eks. hvis man tager et kursus som omhandler estimering af hvor lang tid det tager at udvikle større programmer, eller et teknisk kursus der bygger vider på grundlæggende programmering.

{\Large\textbf{Målgruppe/forudsætninger:}}\newline
Grundlæggende programmering. 

\textbf{Beskrivelse:}\newline
Meget af nutidens viden er spredt rundt over det hele. En søgemaskine gør det muligt, hurtigt at finde det man skal bruge, hvorved der altid skal bruges en specifik søgemaskine.
Søgemaskiner skal bruges i mange af nutidens databaser, da det tager for lang tid at søge bit efter bit. Er det muligt at udvikle en søgemaskine der kan lave en hurtig søgning, og stadig finde de korrekte tegn i flere dokumenter af gangen?
