\chapter{Interview}
\section{Teori}
Der er mange former for interviews og disse kan udføres på forskellige måder, men typisk når der snakkes om interviews, bliver de delt ind i tre forskellige former, enkeltinterview, spørgeskemaer og telefoniske interviews.
Dette afsnit er udarbejdet ud fra PDF'en om dataindsamling, der kan ses i litteraturlisten, under \citep{metodeogprojektskrivning}.\newline

\textbf{Enkeltinterview}\newline
Det enkelte interview forgår på følgende måde: Både intervieweren og respondenten mødes ansigt til ansigt. Fordelene herpå er tydelige, at give respondenten mulig for at besvare private og intime spørgsmål, som en almen respondent ikke er tryg ved at tale om foran andre. Dette mindsker også chancen for at spørgsmålene bliver misforstået, samtidig med at svarene kan diskuteres på et højere plan, end ved et interview over telefonen eller ved et spørgeskema, da snakker med mere end bare ord, nemlig kropssprog. Hvis intervieweren har en god situationsfornemmelse kan et vellykket interview, forventes. 
En udvidelse af enkeltinterviewet kan der snakke om gruppeinterview. Metoden bruges hvis der er pres på tid og ressourcer. Metoden er den samme udover den forskel at der er flere respondenter. Der opfordres ikke til dialog mellem respondenterne. Et modsvar til denne metode er fokusgrupper. Denne metode opfordre netop til dialog mellem respondenterne, men emnet her er temmelig afgrænset. Denne metode bliver brugt til at sammenligne skabelsen af holdninger i sociale miljøer og hvilke argumenter der bliver taget i brug. \newline

\textbf{Telefoninterview}\newline
Det telefoniske interview er lidt en sammenblanding af de to ovennævnte interview former. Telefoninterviewet foregår ved at en eller flere interviewere sidder bag røret og stiller en række spørgsmål, som på forhånd er fastlagte. Den væsentlige forskel på telefoninterviewet som er et kvalitativt interview og spørgeskemaet som er et kvantitativ interview, er at interviewerne kan uddybe deres spørgsmål på et højere plan, end et spørgeskema vil kunne. Måden hvorpå denne form for interview foregår er ved at scanne spørgeguiden ind i et program, hvorefter dette vil blive sendt til respondenten og unødvendige spørgsmål undgås. Her har respondenten så mulighed for at skrive sine egne svar ind, hvilket mindsker fejl ved fx transskription. Til sidst har intervieweren mulighed for at gå i detaljer med hvert spørgsmål sammen med respondenten.\newline

\textbf{Spørgeteknikker og metoder til interview}\newline
Når der snakkes om videnskabelige spørgeteknikker er det vigtigt at kende forskellene på dette og dagligdagssproget, som normalt bliver snakket. Der er de standardiserede spørgeteknikker, hvilket er hvor spørgsmålene og rækkefølgen på disse, omhyggeligt er blevet arbejdet med, og deres rækkefølge, er valgt på forhånd for interviewet. Denne metode er nyttig at tage i brug, hvis en interviewer kender problemstillingen. Her får intervieweren svar på sine spørgsmål med så lidt spildt information som muligt. Dette udføres typisk med spørgeskemaer. Nogle forskere mener, at et standardiseret interview også har det element, at forholdene og endda tiden for interviewet er ens for alle respondenter. Alle andre former for interview er indenfor kategorien ikke-standardiseret interview.
Derudover er der de strukturerede interviews. Dette forgår lidt på samme måde, som de standardiserede interviews. Den væsentligste forskel herpå, er at spørgsmålene ikke er fastlagte, så det kun er spørgeguiden, der er fastsat. Dette skaber større mulighed for en kvalitativ interviewform, hvor intervieweren kan følge op på emner der kommer, som intervieweren ikke havde regnet med. Som et modsvar på denne form for interview, findes det ikke-strukturerede interview. Denne metode har hverken fastlagte spørgsmål, eller en fastlagt spørgeguide/rækkefølge på spørgsmål. Ved brug af denne metode kan intervieweren frit følge et givent emne ud fra respondentens svar, derfor kaldenavnet  ”det fleksible interview”.\newline
 
\textbf{Lukkede og åbne spørgsmål}\newline
Lukkede spørgsmål bruges typisk ved kvantitative spørgeteknikker såsom et spørgeskema. Altså teknikker som gør at responsen let kan sammenlignes og analyseres. Denne metode af spørgsmål falder altså ind under kategorien standardiseret spørgsmål, da respondenten hverken kan ændre på rækkefølgen af spørgsmålene eller gå ind og uddybe sine svar. 
Til hvert et træk, er der et modtræk. De åbne spørgsmål, som bliver benyttet i de kvalitative aspekter indenfor spørgeskemaer, altså den mulighed at respondenterne kan uddybe nogle svar, hvis intervieweren føler det er nødvenligt og stille sådan en plads til rådighed i spørgeskemaet.\newline

\textbf{Psykologiske teknikker}\newline
Interviewerens situationsfornemmelse er kritisk ved at ansigt-til-ansigt interview, altså at intervieweren kan fornemme atmosfæren, hvilke ting intervieweren kan spørge om og hvilke intervieweren ikke kan. At intervieweren kan læse respondentens kropssprog, når spørgsmålene bliver stillet, kan bruges til fordel for interviewet. Hvis dette bliver ignoreret, kan det ende med et mislykket interview.\newline

\textbf{Passive teknikker}\newline
Denne teknik går ud på at stille et spørgsmål, lade respondenten svare, hvorefter intervieweren kommer ind med nogle spørgende kommentar. Ved brug af denne teknik, mindskes interviewerens bestemmelse i retningen af interviewet, og respondenten kan komme med mere information, om et givent emne, og endda indbringe egne meninger og holdninger, hvis dette er vigtigt ift. emnet.\newline

\textbf{Tive teknik}\newline
Hvis respondenten ikke er specielt snakkesalig kan intervieweren give ham/hende følelsen af at den viden de har, er interessant og vigtig, ved fx at spørge, ”Jeg synes dine iagttagelser er interessante” o. lign.\newline

\textbf{Aktiv spørgeteknik}\newline
Hvis intervieweren kommer ud for, at respondenten er meget sky og tilbageholden med information, kan intervieweren manipulere ham/hende til at tro at den information de kommer ud med, er en lille del af en manglende kæde. Fx kan intervieweren nævne en given situation, hvorefter intervieweren spørger ind til det manglende led, altså den information intervieweren mangler. På denne måde ligner respondentens svar ”bare” et lille manglede led i noget intervieweren allerede ved.\newline

\textbf{Interviewet}\newline
Inden vores interview stillede vi nogle spørgsmål op, som der gerne skulle besvares til vores interview med Lars. Disse spørgsmål var skrevet med henvisning af vores underspørgsmål i problemindledning, så vi gerne skulle kunne få noget henblik og svar til vores spørgsmål. Ud over det, ville vi gerne have noget, information omkring turismen i Aalborg, for at se om emnet kunne have relevans for turister i Aalborg. 
Interviewet var sat op til at være et enkeltinterview, og med passive spørgsmål til. Med det ville vi gerne stille vores spørgsmål, hører hvad Lars havde at svare til det, og herefter komme med uddybende spørgsmål, eller få en samtale i gang omkring emnet. Denne teknik blev valgt, fordi at der på den måde måske kunne komme andre ting ind, som gruppen måske ikke havde tænkt på. Så ved denne teknik lader vi Lars tale frit, og komme med alle de ting, han tænker der kunne have relevans for emnet. Vis der i stedet havde brugt lukkede spørgsmål, ville vi kun få dækket det som gruppen havde tænkt på, og måske mistet noget viden Lars havde, der kunne have været relevant for emnet. 
Selve interviewet, forgik meget mere frit, end det var planen. Der blev stillet nogle få af de spørgsmål, som der var lavet på forhånd, men fik ikke stillet dem alle sammen. Lars var rigtig god til at have en samtale i gang, og det slog lidt spørgsmålene til siden, og blev mere en samtale omkring turismen i Aalborg, og Visit Aalborgs situation i turismen. Så interviewet gav et større indblik i turismen i Aalborg, men ikke så mange ideer til løsningsforslag.