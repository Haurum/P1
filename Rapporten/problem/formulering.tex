\chapter{Problemformulering}

I dette afsnit vil den endelige problemformulering blive beskrevet, samt kravene til den endelige løsning. Til sidst afgrænses der i forhold til gruppens evner og de krav der er blevet opstillet. 
Spørgeskemaet bekræftede hypotesen, om at turister har problemer med at finde rundt i storbyer. Dermed blev det bekræftet, at en løsning der kan hjælpe turister med at finde rundt, kan være aktuel. Gennem spørgeskemaet, blev det også klart, at turister helst vil have en interessant rute og opleve byen, i stedet for at komme så hurtigt gennem deres på forhånd valgte attraktioner, som muligt. 


\section{Problemformuleringen}
Hvordan udvikles der en softwareløsning, der hjælper turisten med at finde rundt i en storby, på en interessant rute mellem turistens egne valgte attraktioner?

\section{Krav}
Gruppen har gennem analyser og interviews, fået stillet en række krav til løsningen, af forskellige interessenter. 
Gennem spørgskemaet, blev det konkluderet, at det vigtigste for turister, er at de kan opleve byen på en interessant rute. Derudover har turistbureauet givet udtryk for, at løsningen gerne skal være så simpel som mulig, og løse en specifik opgave godt, da de mener, at det er i turistens bedste interesse. \newline
Der er blevet stillet krav fra universitetes side, om at programmet skal være et lille specifikt program i C, af høj kvalitet. Dette stemmer godt overens, med de krav der er blevet stillet fra turistbureauets side.  

\section{Afgærnsning}
Da dette kun er et P1 projekt, og gruppen er begrænset af både tid, erfaring og diverse krav, har gruppen valgte at begrænse softwareløsningen, på en række punkter. \newline
For det første, bliver programmet begrænset til at finde den hurtigste vej i fugleflugtslinje mellem en række forudbestemte punkter, som turisten selv kan vælge imellem, da det bliver for kompliceret med den viden vi har, at tage højde for veje. Derfor kommer programmet også kun til at kunne fortælle hvilke punkter der er kortest mellem, og ikke hvilken rute der i virkeligheden kommer til at være hurtigst. \newline
Derudover bliver programmet kun lavet på dansk og engelsk, og kommer ikke til at indeholde en grafisk brugerflade, men kommer til at være tekstbaseret.

\section{Løsningsforslag}
På baggrund af kravene og afgrænsningen, har gruppen tænkt sig at lave et program, som har nogle forudbestemte destinationer, der dækker over destinationerne i Aalborg, hvorefter brugeren vælger de destinationer han/hun ønsker at besøge. Programmet vil ud fra disse punkter, beregne den korteste rute, og undersøge om der er andre attraktioner, som ligger tæt på ruten, og spørge brugeren, om det kunne være interessant at besøge disse steder. Hvis ja, vil disse ruter også blive inkluderet. Resultaten bliver en liste over destinationerne, der står i rækkefølge, så turisten ved hvilken rækkefølge de skal besøge dem i, for at få den mest optimale rute.

