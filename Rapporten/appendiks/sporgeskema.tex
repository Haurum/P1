\chapter{Spørgeskema}
I dette appendiks vil teorien til spørgeskemaer blive beskrevet, og rådataen fra respondenterne præsenteret. Teorien er skrevet ud fra pdf'en om dataindsamling, som kan ses under \citep{metodeogprojektskrivning}.
\section{Teori}
Spørgeskemaer er en form for interview, hvor den kvantitative metode benyttes, som kan ske både online og offline. I dette projekt er den online løsning blevet benyttet, hvilket kaldes et internetinterview. Den offline metode kaldes for postale undersøgelser. Denne metode foregår ved at respondenten får spørgeskemaet fysik i papirformat. Fordelen ved internetinterview er den hurtige respons fra respondenterne og den gør resultatbehandlingen nemmere. [1]\newline

Når et spørgeskema skal udføres er det vigtigt at have helt styr på formålet, altså hvad er succes kriterierne, hvilke problemstillinger skal der være svaret på, efter spørgeskemaet er fuldført? Den anden ting som er vigtig at få afklaret, er hvem målgruppen for undersøgelsen er, hvem er det der skal svare på disse spørgsmål? Med disse to ting i baghovedet skal et spørgeskema så udføres, så formålet bliver opfyldt så godt som muligt, uden at gøre spørgeskemaet forvirrende for målgruppen.

\subsection{Spørgsmålene}
Spørgeskemaets spørgsmål skal først og fremmest være dækkende, på den måde at den skal kunne dække alle problemstillingerne som er blevet udformet. Derudover skal disse spørgsmål ikke gå ud over de opstilte problemstillingerne. Et spørgsmål som ikke svarer på en del af problemstillingen, er ikke brugbar i den efterfølgende analyse. Spørgsmålene skal være lige til sagen, der er ingen grund til for mange omveje. Som hovedregel er korte spørgsmål bedre end lange, da disse er mere direkte og overskuelige. 
\subsection{Formulering}
Det er vigtigt i forbindelse med formuleringen af spørgsmålene til spørgeskemaet, at alle spørgsmålene vil blive forstået ens. Der må altså ikke være nogen tvivl hos respondenten. Det er i denne forbindelse vigtigt ikke at bruge vage formuleringer, dobbeltspørgsmål, indforstået/faglig jargon og lange ord. Spørgsmålene kan give forskellige meninger for forskellige respondanter hvis disse ikke bliver overholdt, og dermed kan svarene ikke bruges til ret meget analytisk.
Til sidst er ledende spørgsmål også farlige, da spørgeskemaundersøgelser som oftest har som mål at give objektive ikke forvrængede resultater.


\subsection{Rækkefølge}
Når rækkefølgen skal overvejes er der 4 ting man skal tage højde for: 
-Motivationen hos respondenten
Respondenten vil efterhånden miste koncentrationen og motivationen til at besvare spørgeskemaet jo længere han eller hun kommer. Dette kan undgås ved først og fremmest kun at stille de nødvendige spørgsmål så spørgeskemaet bliver så kort som muligt. Derudover kan det være en fordel at stille de nemme spørgsmål i starten af spørgeskemaet så respondenten får besvaret en masse spørgsmål i en fart, og derved kommer godt i gang. Det kan dog også være en god ide i nogle tilfælde at komme til sagen med det samme.
-Konteksten spørgsmålet er indenfor 
I nogle tilfælde kræves der information fra et tidligere spørgsmål, før der kan svares på et andet. Derfor er det vigtigt at sørge for at rækkefælgen er således at respondenten har fået stillet de krævede spørgsmål før det spørgsmål hvor respondenten skal bruge informationen.
-Den emnemæssige sammenhæng 
I rækkefølgen på spørgeskemaet skal der undgås alt for store emnemæssige brud, der skal været et naturligt flow mellem spørgsmålene.
-Tragtmodellen
Det er til stor fordel at stille de generelle spørgsmål før de specifikke, så der opstår en form for tragtmodel ned gennem spørgeskemaet.


\section{Rådata}
Antal respondenter: 60

\textbf{Hvad er vigtigt for dig på din storbyferie?
Sæt gerne flere krydser}

    \begin{tabular}{| l | l |}
    \hline
    Svarvalg & Besvarelser \\ \hline
    Opleve kulturen & 68.33\% - 41 \\ \hline
    Se byens seværdigheder & 90\% - 54 \\ \hline
    Shopping & 50\% - 30 \\ \hline
    Fest/Bytur & 26.67\% - 16 \\ \hline
    Museumbesøg & 11.67\% - 7 \\ \hline
    Maden & 63.33\% - 38 \\ \hline
    Teater/Musik & 18.33\% - 11 \\
    \hline
    \end{tabular}


Kommentarer:
Sportsbegivenheder\newline

\textbf{I hvilken grad bliver din storbyferie planlagt?}

    \begin{tabular}{| l | l |}
    \hline
    Svarvalg & Besvarelser \\ \hline
    Alt er planlagt til punkt og prikke inden ferie & 0\% - 0 \\ \hline
    Nogle ting er planlagt inden ferien & 70\% - 42 \\ \hline
    Planlægger dag for dag på ferien & 28.33\% - 17 \\ \hline
    Planlægger ikke & 1.67\% 1.67 - 1 \\
    \hline
    \end{tabular}
\newline
\newline

\textbf{Hvilke hjælpemidler bruger du til at planlægge din storbysferie?}

\begin{itemize}
	\item Tripadvisor
	\item Google
	\item Internettet
	\item Guide
	\item Google Maps
	\item Yelp
	\item Turen går til..
	\item Venner/familie som har været på stedet
	\item Rejsebureauer
	\item Bøger
	\item Lonely planet
	\item fdm-travel.dk
	\item booking.com
	\item hotels.com
	\item Momondo
	\item Ansrejser
	\item Expedia
	\item Hertz biludlejling
	\item Scout
	\item Lokale
	\item Kort
	\item Politikens de røde
	\item Top 10 vigtigste ting at se\newline
	
\end{itemize}

<<<<<<< HEAD
Hvilke redskaber bruger du til at finde rundt når du er på storbyferie?
Vælg gerne flere

Svarvalg - Besvarelser\newline
De lokale:
46.67\%  -  28 \newline
Diverse kort/brochure:
81.67\%  -  49 \newline
GPS/Elektronisk kort:
55.00\%  -  33 \newline
Taxa:
10.00\%  -   6 \newline
Guider:
18.33\% - 11 \newline


Har du nogensinde haft problemer med at finde vej på din storbyferie?

Svarvalg - Besvarelser \newline
Ja: 
68.33\%  -  41 \newline
Nej
31.67\%  -  19

Når du skal fra en aktivitet til en anden på din storbyferie, vil du helst tage den hurtigste rute eller en langsommere men mere interessant rute?

Svarvalg - Besvarelser \newline
Den hurtigste rute:
20.00\%  -  12 \newline
Den interessante rute
80.00\%  -  48 

Et program/applikation, som hjælper mig med at finde den hurtigste og/eller mest interessante vej igennem byen, via mene valgte ”must see” destinationer, ville være noget jeg kunne bruge? \newline
=======

\textbf{Hvilke redskaber bruger du til at finde rundt når du er på storbyferie?
Vælg gerne flere}

	\begin{tabular}{| l | l |}
	\hline
	Svarvalg & Besvarelser \\ \hline
	De lokale & 46.67\% - 28 \\ \hline
	Diverse kort/Brochure & 81.67\% - 49 \\ \hline
	GPS/Elektronisk kort & 55\% - 33 \\ \hline
	Taxa & 10\% - 6 \\ \hline
	Guider & 18.33\% -11 \\
	\hline
	\end{tabular}
\newline
\newline 	


\textbf{Har du nogensinde haft problemer med at finde vej på din storbyferie?}

	\begin{tabular}{| l | l |}
	\hline
	Svarvalg & Besvarelser \\ \hline
	Ja & 68.33\% - 41 \\ \hline
	Nej & 31.67\% - 19 \\
	\hline
	\end{tabular}
\newline
\newline

\textbf{Når du skal fra en aktivitet til en anden på din storbyferie, vil du helst tage den hurtigste rute eller en langsommere men mere interessant rute?}

	\begin{tabular}{| l | l |}
	\hline
	Svarvalg & Besvarelser \\ \hline
	Den hurtige rute & 20\% - 12 \\ \hline
	Den interessante rute & 80\% - 48 \\
	\hline
	\end{tabular}
\newline
\newline

\textbf{Et program/applikation, som hjælper mig med at finde den hurtigste og/eller mest interessante vej igennem byen, via mene valgte ”must see” destinationer, ville være noget jeg kunne bruge?}
>>>>>>> origin/master

\begin{center}
	\begin{tabular}{| l | l | l | l |}
    \hline
    Meget uenig & Uenig & Enig & Meget enig \\ \hline
    1.67\% - 1 & 6.67\% - 4 & 45\% - 27 & 46.67\% - 28 \\
    \hline
    \end{tabular}
\end{center}
