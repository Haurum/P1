\chapter{Diskussion}
I dette kapitel vil gruppen opsamle på de beslutninger, som er blevet truffet gennem projektet. Samtidig vil gruppen se på fejlkilder der kunne være opstået og hvilken indflydelse disse fejlkilder har kunne påvirke projektet. 

\section{Spørgeskema}
Tidligt i projektet sendte gruppen et spørgeskema ud på deres facebookprofiler, som i alt gav 60 besvarelser, og gennemsnits alderen ville være relativ lav. Gruppen havde en række spørgsmål som vi gerne ville have svar på, og derfor blev spørgeskemaet opbygget af mange underspørgsmål, som var blevet lavet i forhold til den initierende problemstilling. Spørgeskemaet konstaterede nogle problemer, hvor det nok var forudsigeligt hvad respondenterne ville svare på de stillede spørgsmål, dette var bl.a. om turisterne havde et problem med at finde rundt på deres storbyferie.

Turister var sat som gidsler i interessentanalysen, men gruppen valgte at bruge dem som en delvis ressourceperson, da det var dem som programmet var rettet imod. Dog mener gruppen at spørgeskemaet blev sendt for hurtigt ud, og uden eftertanke. Med det menes det at mange af spørgsmålene ikke var formuleret ordenligt, og man let kunne gætte sig til hvad respondenterne ville svare på spørgsmålene, dermed at mange af spørgsmålene var meget ledende. Selve projektet er rettet mod turister i Aalborg, hvor 41\% af turisterne er udenlandske, så da gruppen havde valgt at skrive spørgeskemaet på dansk, rammes hele målgruppen ikke. Gruppen havde som sagt delt spørgeskemaet på deres facebookprofiler, hvilket også begrænser respondentgruppen, dog vil de fleste være dansktalende, hvilket i denne situation ikke gjorde så meget at spørgeskemaet så var skrevet på dansk. For at få flere respondenter, og specielt udenlandske respondenter, burde gruppen have delt spørgeskemaet flere steder end bare på Facebook, samt lavet spørgeskemaet på både engelsk og dansk. 

Når gruppen reflektere over dette projektforløb, ville vi gerne have brugt mere energi til på dette spørgeskema, og med eftertanke burde vi nok have udsendt et nyt spørgeskema. Selve spørgsmålene skulle være bedre gennemtænk og ikke så ledende som de var blevet skrevet. Ud fra spørgeskemaet blev der dog konstateret et problem, som projektet kunne tage udgangspunkt i. En respondent gruppe er dog relativ lille, og det ville have været bedre at få en både større men også bredere respondent gruppe. Med ordet bredere mener gruppen at der ønskes både respondenter fra ind- og udland i forskellige aldre.

\section{Interview}
I dette projekts interessentanalyse blev turistbureauer, i dette tilfælde VisitAalborg, sat som ressourceperson, da de kunne give gruppen en del informationer om turisme i Aalborg. Derfor besluttede gruppen at prøve at skaffe et interview med en medarbejder fra VisitAalborg, hvor gruppen så fik fat i Lars Bech (og Kim Mikael Jensen), som gerne ville stille op til et interview. Gruppen havde udarbejdet en interviewguide, hvilket var lavet i punktform, som beskrev hvad vi ville fortælle om vores emne, og hvilke spørgsmål vi gerne ville stille til Lars. Interviewet var planlagt til at det skulle udføres som et ustruktureret interview, hvilket vil betyde at spørgsmålenes rækkefølge ikke er fastlagte, hvilket vil gøre et interview mere fleksibelt. Hvis Lars ikke var meget for at snakke, eller ikke kom frem med lige det vi kiggede efter, ville vi kunne spørge mere ind til emnet. Dette var dog ikke tilfældet med Lars, han snakkede rigtig meget, hvor han ofte førte interviewet videre. Da vi i gruppen ikke rigtig havde lavet interview før, var interviewerne ikke så gode til at stoppe ham, når han snakkede videre end de stillede spørgsmålene. Dette gjorde at interviewet udviklede sig til, at Lars nærmest tog styringen af interviewet.

Ud fra interviewet med Lars, blev gruppen klogere på hvilke turister og hvilke attraktioner der er populære i Aalborg. Lars og Kim virkede interesserede i projektet, de havde dog tidligere arbejdet med en elektronisk løsning, men var blevet nødsaget til at gå tilbage til kort og brochure. 

Interviewet var lige som spørgeskemaet, var udført uden den store eftertanke. Dette var skyld i, at interviewguiden ikke blev så god, som gruppen havde håbet. Gruppen skulle have ventet til lidt længere i forløbet, så gruppen havde mere konkret viden om emnet og om hvilke spørgsmål gruppen ville spørge professionelle på emnet om. 

\textbf{Dette er ikke færdigt endnu} 
\section{Programmet}
Programmet er lavet som beskrevet i afsnittet ”Implementering”. Gruppen havde helt fra begyndelsen valgt at lave løsningen i fugleflugtslinje, og der er derved ikke taget højde for vejnettet i Aalborg. Dette har simplificeret programmet, men også givet en usikkerhed, når den interessent skal bestemmes. Gruppen kan ikke garantere, at den givne rute i realiteten er den hurtigste, når der også skal tages højde for hvilke veje man rent faktisk kan bevæge sig på og kan komme igennem. 
Hvis gruppen skulle have implementeret en løsning, der tager højde for vejnettet, havde gruppen tænkt på to forskellige løsninger. Den første ville være at sætte hele Aalborg op i et grid, hvor vejnettet ville blive markeret med 1 og resten ville markeret med 0. På den måde ville vejen kunne findes med forskellige søgealgoritmer fx A*. En anden løsning ville være at have en tabel, der indikerede hvilke veje der er forbundet og distancen der i mellem. Disse løsninger ville have gjort ruten mere præcis, da den reelle korteste rute ville kunne findes. 

\section{Algoritmer}
For at beregne den korteste rute mellem attraktionerne, valgte vi i gruppen at benytte Nearest Neighbour Algoritm, på grund af dens hurtige eksekveringstid, der muliggøre det for brugeren af programmet, at indtaste et højt antal attraktioner, uden at det går mærkbart ud over oplevelsen med programmet. Algoritmen blev også valgt, på grund af den forholdsvis simple implementering af den, og at generelt passede godt til vores behov. Problemet ved at bruge denne algoritme, er at den ikke nødvendigvis finder den hurtigste rute. Algoritmen er upræcis, og en hvis fejlmargen bliver nødt til at accepteres, hvis ikke man vælger at gribe ind overfor algoritmen, hvis den er ved at gøre noget der tydeligt giver en længere rute end nødvendigt. Den korteste rute burde fx aldrig krydse sig selv, og det var muligvis en af de ting vi kunne have tjekket for, når algoritmen benyttes, for at sikre os at den i det mindste ikke gør det, og på den måde får en kortere rute, end hvis vi bare havde ladet den gennemføre sine beregninger. 

En anden løsning, ville være at prøve alle ruter der overhovedet er for de valgte attraktioner, men eksekveringstiden stiger faktorielt med antallet af attraktioner, så der skal ikke vælges mere end et par stykker, før brugeren af programmet begynder at kunne mærke at det tager lang tid at lave beregningerne. Et scenarie vi helst gerne ville undgå.
Double Minimum Spanning Tree og Djikstras var også algoritmer gruppen havde undersøgt, og prøvet at implementere, men som beskrevet i afsnittet om grafteori, så var det ikke optimalt til de behov der var for programmet. 

